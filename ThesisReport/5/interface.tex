% this file is called up by thesis.tex
% content in this file will be fed into the main document

%: ----------------------- name of chapter  -------------------------
\chapter{Windowing Toolkit} % top level followed by section, subsection


%: ----------------------- paths to graphics ------------------------

% change according to folder and file names
\ifpdf
    \graphicspath{{5/figures/PNG/}{5/figures/PDF/}{5/figures/}}
\else
    \graphicspath{{5/figures/EPS/}{5/figures/}}
\fi

%: ----------------------- contents from here ------------------------

\section{Motivation}

In order to improve stability of the system, a windowing toolkit was developed to display information to the user. Interactions with the user occur in only a few pre-definable ways. The user will either view data that the system has produced, or press buttons on the screen in order transition the display to another screen. The currently developed software (EyeLin) provided very low level access to complete these actions. This allows for greater flexibility in developing using the software, but makes it  more confusing to deal with, and contributes to bugs in the software. In order to alleviate this problem, a high-level abstraction was developed on top of the existing software. This abstraction allows the low level functionality, such as interacting with the touchscreen, to be hidden. This simplifies development when layout the user interface. This also allows for code to be written and debugged once, for example the toolkit completely removes interactions with the touchscreen, allowing buttons to be added in a much simpler fashion.

As other requirements of the system required C++ libraries, the windowing toolkit was implemented using C++. This is an object oriented language that provides a few distinct advantages over the C language that is traditionally used on embedded devices such as this. The first advantage is the use of object orientation. When constructing a user interface, it is much easier to understand the different elements of the interface as unique objects. By thinking of a text display as a single object, it becomes more intuitive to manipulate. The other advantage in the use of C++ is that it allows for objects to inherit from other objects. This use of polymorphism allows elements to implement the functionality they require by inheriting from specifications in the toolkit. The toolkit is able to call these new functions due to use of virtual methods.

The toolkit has been written in a way that the programmer utilizing it does not need to understand anything about threads. While the mechanics inside the toolkit do use threads, the interaction with this threads is completely abstracted away. It is not possible to directly manipulate the underlying threads with the toolkit. This removes any problems with synchronization between threads, as they cannot be manipulated. There are three threads running inside the toolkit. One thread is responsible for manipulating the screen, one thread is responsible for responding too events occurring on the touchscreen  and the last thread is responsible for reading and processing the network messages.

The final advantage of developing this toolkit, is that it is not limited to use in this project. As the toolkit is written in a generic way, with the exception of being able to receive the messages transmitted over the network, it is able to be deployed on future projects. This allows future projects to have a rich user interface, while keeping the high-level abstraction in place.

\section{Elements}

The windowing toolkit consists of pre-made classes that the are either used directly, such as the digit display element, or are inherited from, such as the base or runnable classes. Figure \ref{uiframework3} shows the classes developed in the toolkit.

\figuremacro{uiframework3}{UML diagram of the window toolkit}{}

\subsection{UIElement}

The UIElement is the most basic class definition in the toolkit. It is an abstract class that can never be instantiated. Its purpose is to define methods for interacting with the screen and other objects. It also provides default functionality for most methods, this allows all the classes that inherit from it to function the same way.

\subsubsection{addChild()}

An important definition of the UIElement class, is that it may contain any number of other UIElement classes inside it. To add another UIElement class the method addChild() is called. This stores the child element inside a C++ vector whose length is only limited by the amount of ram inside the machine. The advantage of this is that actions can be performed on a UIElement and all it's children. If a element needs to be drawn, all it's children will be drawn too or if an element is disabled, all it's children will be disabled too.

\subsubsection{draw()}

 The most important method of the UIElement class is the draw method.  This method definition is not implemented in the UIElement class, it is defined a virtual abstract method. All classes that can be instantiated must implement this function call. The purpose of this function call is to allow the user to specify the low level commands that are used to display this element. This can include drawing lines, squares, or setting individual pixels. This method should be implemented by the programmer, but should only ever be called by the mechanics of the toolkit. The developed of the user interface should never directly call this method.

\subsubsection{enqueueDraw()}

enqueueDraw() is called whenever the system, or the programmer, wants to trigger a refresh of the screen. This will signal the toolkit that a redraw should be prepared. The purpose of this method is two-fold. Firstly it allows the toolkit to perform optimizations of the draw function in order to maximize speed (see \ref{Redrawing the Screen}). Secondly it's implementation has O(1) complexity. This means that the call to enqueueDraw() completes in constant time. This is used for performance reasons, as whatever thread has called enqueueDraw(), will not need to wait for the screen to be redrawed, it will return instantly. The draw will be scheduled to occur some time after enqueueDraw() is called. By default this method will also call the enqueueDraw() method of all it's children, allowing entire sections of the display to be redrawn using one function call.

\subsubsection{animate()}
TODO

\subsubsection{setActive()}

A property that is required of any draw-able object inside the toolkit is whether it is current being displayed to the user. This property allows elements to exist in the machines memory, but only be draw if they are current being displayed. To manipulate this property, the method setActive() is called. This method allows the state to be set to either true or false, meaning that the object will be drawn or not drawn respectively. If the active state is false, calls to enqueueDraw() will be processed, but the call to the draw() function will be skipped. Thus individual elements of the display can be hidden at will. This method will also call the setActive() method of all the children of this element. Thus allowing sections of the display to be hidden with one function call.

\subsubsection{isActive()}

This method will return the current state of the element. This is used to check whether the current element is being drawn or not. This method is called internally by the screen drawing mechanics. It can also be used in order to check whether the element is being displayed, and perform different tasks if it is not being displayed.

\subsection{base}

The base element represents the panels or windows that are being displayed to the user. An important property of the base element is that it is defined to occupy the whole screen. This element will draw the entire width of the screen, which will clear any old draws that may still be present. Another extended property of the base element is that it maintains a list of all the other base elements that are present. This is used in order to allow for global navigation buttons. Rather than layout buttons in the same location on every screen, a button can be added as a global button. This will ensure that it appears on all the screens present in the list. This allows the buttons and their location on the screen to be defined once, making the final program more stable and simpler to understand.
For further discussion on the button element see \ref{Button}.
The base class itself is abstract, it cannot be instansiated. There is no way to display a "default" base element. In order to build a panel, the panel must inherit from the base element, and implement at minimum the abstract function getButton().

\subsubsection{base()}

The constructor for this class takes a single argument. This argument is a boolean value indiciating whether this panel will display global buttons or not. By default this option is set to true, though it can be easily overriden when a new class is inheriting from this one. This optional argument allows a panel to forego global buttons. This is used in the case when the developed only wishes to display a simpler panel, or when the current set of global buttons would appear in-appropriately on this panel.

\subsubsection{draw()}

This class implements a rudimentary version of the draw method. This specifies a default display for each fullscreen panel the user will view. The reasons for this functionality are two-fold. Firstly, it allows a common backdrop to be defined for each screen. This keeps a sense of consitency while navigating, as the background will always be similar. It is also useful in speeding up development, as having a default state for a panel is useful for prototyping and implementing new panels. The second reason a default is specified is to remove any elements that were present on the previous screen. Even though buttons and UIelements may not be active anymore, pixels may still be set corresponding to their images. These pixels may have last been refreshed a long time ago. They will remain in the framebuffer until they are overwritten by a new set of image data. As the a base panel is defined as occupying the whole screen, it will replace and UIElements that may already exist on the screen, thus removing them from being visible. As the instance of this base class with be enqueued onto the drawing queue first, all it's child elements will still be drawn correctly.

\subsubsection{getButton()}

In order to transition to a panel, an action must be undertaken by either the user or the system. The most common way of transisitioning would be when the user wants to display a different screen. Typically this would occur by the use of pressing buttons. This is why any base panel must implement the getButton() function. This function returns a button object that contains the image data to display for this button, and the action to undertake when the button is pressed. This action will typically be a call to the activate() function of the panel, though other actions can be called before the call to activate().

\subsubsection{addButton()}

A common element that will be placed on any panel is a button element. This is the basic way in which the user navigates. A button element itself is a child element of the panel it is contained in, it needs to be drawn when the panel is drawn. A button will also have extra functionality than just being drawn, namely, it can be pressed by the user. As the existing addChild() function only adds the element to the list of child elements to be drawn, an extra method was developed to add buttons to a panel. This method is the addButton method. This method adds the button element to the list of children, but also registers the button element as an interactable object. Internally this is achieved by adding the button to a list that contains only buttons. This list is ordered in the order that the buttons were added to the panel. Maintain this list is important in order to translate the lower level screen interactions into finding which button was pressed see \ref{button_thread}.

\subsubsection{addGlobalButton()}

Mentioned earlier was the use of global buttons, which are buttons that will appear on every class that inherits from base. In order to distinguish between buttons that are added to every base panel and buttons that are added only to the current base panel, the addGlobalButton() function was developed. This method is declared statically and does not need to be called on an instance of a base class, however it does need at least one instance to have been created for it's effects to be observable. In it's simplest form, this method iterates over the list of base elements, and calls the addButton() method on each base using the supplied button argument. This allows the same button to appear on multiple screens, due to the fact that it is the exact same object, it will perform the exact same action and be laid out in the exact same place on each screen.

\subsubsection{refreshTouchMap()}

This method is an internal method to the framework, and should not need to be called by the developer. It is responsible for setting up the touch screen in the lower level libraries. It takes the list of buttons that has been built by calling addButton() and registers each button and the region the button exists on with the touch drivers. This abstracts any interaction or understanding of how the touch screen mechanics away from the developer. This function will be called whenever activate is called, thus setting up the framework to respond to actions on the buttons that exist on the current panel. It will also clear the previously registered buttons, so they cannot be clicked while the system is displaying the new panel.

\subsubsection{buttonPressed()}

The buttonPressed() method defines what happens when the user presses a button. This method is implemented in the base class definition in a way that it should never need to be overwritten in any classes inheriting from the base class. This method takes the position in the list of buttons that is pressed and performs the actions that occur when the button is pressed. The runnable abstract class allows any action to be coded and run by this method see \ref{runnable}. This method also performs things like animating the buttons. In order to provide instant feedback to the user pressing a button, the framework will invert the colour of the button. After a short while, the button will turn back to it's previous state, and the buttons action will be performed. This small animation provides instant feedback that the user has pressed the button, and results in a much more enjoyable user experience. Delaying the action also has another advantage. By waiting a short time between running the action, which is usually the display of a different base panel, the framework is able to remove and duplicate keypresses that may occur. This prevents the user from pressing a button to transisition into a panel, and then immediately pressing a button in the new panel which they did not intend to press.

\subsubsection{activate()}

The activate method is defined in the base class. It can be overwritten in classes that inherit from the base class, though any class implementing different functionality should call the base classes activate() method as well. This method's main responsibility is to display or 'activate' the panel that it is called upon, and to translate the touch driver information into button presses. The default functionality of this method, is to call enqueueDraw() first. This will display the current panel on the screen. Next it will call refreshTouchMap(), to register the active buttons to the framework. Once the panel has been setup, this method will then block and await input from the user via screen events. Thus this method should only ever be called in the thread that is responsible for controlling the button presses. A apparent downside to this is that there is no easy to for the developer to change the active panel inside a different thread, it is certainly possible to change the active panel though it is more difficult than changing in response to user interaction. On the other hand, this limitation is actually advantageous from a user friendliness standpoint. As the screen will only change from different base panels in response to actions performed by the user, it is easier for them to mentally link actions they have performed into re-actions displayed by the device. This specification means that the panels won't transisition by themselves, leading to a much simpler and easier to understand user interface. 

This method also allows animations to be specified. By supplying an argument defining the type of screen transition to use, activate method will perform the actions necessary to animate the transisition between the previous panel and this one. These transisitions are implemented inside the screen driver, and include sliding the screen in and out. For performance reasons, the default transistion is an in-place swap, which is the most efficient in respect to CPU time.

When this method finishes it returns the next action to run. This could either be a class that runs the next panel, or some other action to be performed when this panel is not longer being displayed.

\subsection{runnable}

The runnable class is an abstract class which is used to allow the developer using the toolkit to run actions inside the toolkit. By using polymorphism, the toolkit is able to call methods on classes which were not programmed or compiled with the toolkit. This runnable aspect is used in order to 'run' various aspects of the user interface. For instance, the base class's activate function must return a pointer to a runnable class. This is used to define what happens when the panel is no longer active. Runnable classes are also used extensively by the button system. A button must contain a runnable class. This class will be run when the button is pressed. Thus allowing any code to run whenever a button is pressed by the user. 

\subsubsection{isScreenChange}

A property used in the runnable class is whether this action is triggering a screen change or not. This property is set to true by default. If this property is to, the previous base panel that called this action will have it's active property set to false. This suppresses the last screen from being changed, so the runnable action does not need to know where it was called from. If this property is set to false, the previous panel will not be flagged as inactive. This is used whenever an action is updating a element on the user interface, and does only wants to trigger a redraw of the element that changed. The previous panel will remain active, and will resume listening to touch events when the runnable action is finished.

\subsubsection{run()}

This is the abstract method that must be implemented in any class inheriting from the runnable class.




\section{Subscriber}

\section{Button Loop}

\section{Message Queue}

%
%: ----------------------- contents from here ------------------------

\section{Battery}


\figuremacro{batt}{Panel showing other panels}{Shortcuts to different aspects of the program. (Battery, Maps, Trip Meter, Accelerometer, Arduino, Savings, About, Options)}



% ---------------------------------------------------------------------------
%: ----------------------- end of thesis sub-document ------------------------
% ---------------------------------------------------------------------------


%
%: ----------------------- contents from here ------------------------

\section{Maps}

A common, yet useful driver aid is displaying the map of the current location. This is what the maps panel does. It listens to the GPS messages to determine the position of the car. A screen-shot of this panel is shown in Figure \ref{map}. This panel uses the full area to display the current location of the vehicle on the screen. The actual position of the vehicle is centered on the middle of the screen, allowing the driver to view streets and landmarks relative to his current position. The panel features several levels of zoom, controlled by the bar on the top right corner on the screen. The plus button will zoom in, and move the slider to the right, while the minus button will zoom out, and move the slider to the left. The system supports various levels of zoom, being able to display a few buildings relative to the car or being able to display the surrounding suburbs.

\figuremacro{map}{Map display panel}{Screen-shot of the map display panel, showing the map of the current location and the map controls visible}

\subsection{Map Data}

In order to display the map images, the map data must first be obtained and stored. There are many possible methods for doing this, ranging from creating the maps as needed, or downloading them from external services and storing them in a cache. The relatively low CPU power of the eye-bot m6 makes rendering the maps on-the-fly a undesirable prospect. Open source map data for the entire planet results in a file that is 18GB in size \cite{planet_osm}. This file is much too large to store locally, and this file is actually storing a compressed version of the data. Even if it were possible for the eye-bot to store this file, via the use of external storage, it would be a large strain on the CPU to convert the street level data into viewable maps. It would also require many other pieces of software to be installed on the device, making it much more complicated to manage.

Another possibility is to use an existing Internet based map server and download the map imagery as needed. The has several downsides. Foremost it requires an Internet connection whenever to display the maps. The system does have a 3G connection installed, but this cannot be considered a dependable communication channel. It is highly likely to drop out, and is limited in coverage to the areas in which it has reception. Even without these issues, most map-servers do not allow you to download maps in bulk, as this violates their usage policies \cite{tile_usage_policy}. This makes this method undesirable as it is not suit-able for downloading maps in bulk, or as needed. Attempting to pre-download all these maps using a non-3g link would result in a violation of the usage policy.

The method chosen to obtain the map data was to pre-create create the map data using a more powerful machine. A map-server was setup and loaded with all the street data for the oceania region. For more information on the map-server setup see appendix \ref{app:mapserver}. This method overcomes the problems of the previously mentioned methods. All the processing is done on the much faster machine in advance. The area processed in advance is defined by the properties in table \ref{tab:mapbbox}. This area is depicted by the image shown in Figure \ref{mapperth}. The expanse of these maps covers all of metropolitan Perth. In future more maps can easily be processed, however this will result in more storage space being required. The current settings are a good balance between storage and expanse of data. This is because the map data will be less useful outside of the city, and the car is typically not driven any further than the pre-rendered maps. 

\begin{table}
\begin{center}
	\label{tab:mapbbox} 
   \begin{tabular}{|l|l|}
        \hline
        Property           & Value           \\ \hline
        Minimum Zoom Level & 11              \\ 
        Maximum Zoom Level & 18              \\ 
        Top Left           & 115.687,-31.71  \\ 
        Bottom Right       & 116.508,-32.253 \\
        \hline
    \end{tabular}
\end{center}
\end{table}

\figuremacro{mapperth}{Pre-rendered map size}{This figure shows the area that has been pre-rendered for use in the map panel display}


% ---------------------------------------------------------------------------
%: ----------------------- end of thesis sub-document ------------------------
% ---------------------------------------------------------------------------


%
%: ----------------------- contents from here ------------------------

\section{Trip Meter}

A useful driver aid that is common on vehicles is that of a trip meter. Traditionally this component records the distance the car has traveled since the trip meter was set. This functionality is usually a result of the simple systems in place, and is tied to the revolutions of the wheels on the vehicle. As this system has more hardware at it's disposal, the trip meter can implement more functionality than a standard trip meter, making it much more useful in examining the performance of the car. Figure \ref{trip} shows the trip meter panel being displayed on the screen. An important note of this panel is that all calculations are done whether the panel is being displayed or not.

\figuremacro{trip}{Trip Meter Panel}{This figure shows two independent trip meters and the best record speed data}

The first unique point of this trip meter, is that it contains two independent meters. This is useful as it allows the driver to evaluate the statistics of two overlapping trips. One trip meter can be used to record the distance traveled since the car was last charged, while the other can be used to record the distance traveled in the last week or month. This independence allows the operator to decide how best to use the trip meter data, resulting in a high level of flexibility. In figure \ref{trip} the trip meters are located on the left, sitting above each other.

Each trip meter records the distance traveled, the time elapsed since the meter was started, the time the car has been moving since the meter was started and calculations based on the elapsed and moving time. These statistics are displayed live to the user, but are not logged, as the logging functionality  is taken car of by a different component in the system. The trip meter panel also displays the current moving speed in the top right. Below the moving speed are the best run records in seconds. This allows the driver to have quick feedback as to how the car is performing, without having to do lots of processing on logged data.

\subsection{Distance Driven}
\label{sec:distancedriven}
The most important part of a trip meter, is the distance that the meter has recorded. This is shown on figure \ref{trip} at the bottom of each trip meter. The meter stores the distance driven internally as a double length floating point number, but displays it on the screen as a rounded integer. This is done in order improve precision for later calculations as other values will depend on the distance that has been driven. Equation \ref{eq:distancedriven} shows the the distance is calculated based on the GPS position.

\begin{align}
\label{eq:distancedriven}
\mathrm{Distance} &=\mathrm{Distance}_\mathrm{lastrun} + \mathrm{Speed}(\mathrm{Time}_\mathrm{now}-\mathrm{Time}_\mathrm{lastrun})
\end{align}

The method for working out is a continuous function that is based on the last known distance the car has traveled. This method was chosen as it does not require any information other than the last time the formula was run, and the last distance calculated. This also makes the trip meter flexible in that the distance calculation does not need to be processed at exact intervals. If messages are dropped for whatever reason, the calculation will still take place, though it will not be as accurate as it could be. The calculation will be able to cope with fluctuations in the message timing, and can adapt to the speed of the GPS being used.

The downside of this method, is that big changes in time can cause problems with the calculation. If the signal drops out for an extended period of time, such as going through a tunnel, the calculation in \ref{eq:distancedriven} would have a big margin for error. In order to prevent this, the trip meter will ignore large time differences. If two calculations are over 10 seconds apart, the result will not be trusted, and not be used in the calculation. This prevents GPS signal loss from having an adverse effect on the trip meter calculations, but does impose a limit on the trip meter.

The limitation of this method of calculating the distance driven is that it relies on the GPS messages being sent to it. If the GPS signal is lost, the distance driven will not be increased. This is a limitation imposed by the use of the GPS sensor, and cannot be avoided, as attempting to guess the distance driven while the GPS signal has been lost has a very high probability of being incorrect.

\subsection{Time Elapsed}

Another variable displayed on the trip panel is the time elapsed. This is simply the time elapsed since the trip meter was last reset. This time increments even while the GPS signal has been lost, performing a stop-watch like action on the trip. While it may seem natural to just record the time that the trip meter was started and subtract it from the current system time, this would lead to problems during the system startup. The time needs to increment even while the GPS is connecting, and must be resistant to changes in the systems internal clock. Thus the time is calculated similar to section \ref{sec:distancedriven}. The formula used to calculate the elapsed time is given in equation \ref{eq:timeelapsed}. The time elapsed is displayed as the highest element of the trip meter in Figure \ref{trip}

\begin{align}
\label{eq:timeelapsed}
\mathrm{TimeElapsed} &=\mathrm{TimeElapsed}_\mathrm{lastrun} + (\mathrm{Time}_\mathrm{now}-\mathrm{Time}_\mathrm{lastrun})
\end{align}

Much like the distance, this method of calculation depends on the last known values. This means it does not matter at the actual time the system trip started once the timer has been running. This makes it resistant to changes in the system time, and thus makes the timer more robust. 
This timer records the time elapsed on the nanosecond level, as it is used elsewhere in calculations. For display, the timer converts these values into the traditional hours, minutes, seconds format that is easy for the operator to read.

\subsection{ Moving Time}

An aspect of the cars telemetry that would be interesting to the driver is the cars moving time. This is defined as the time in which the car has spent in motion. The main use of this data point is to contrast it against the elapsed time, to highlight how long the car has spent sitting still in traffic. This variable is also useful to record for future calculations, such as working out the average speed of the trip. This element is calculated according to equation \ref{eq:timeelapsed}, except that it will not update if the current speed of the car is 0~km/h. As such this element requires the speed of the car to be processed, so it cannot be calculated when the GPS signal is lost. This fits in with the functionality defined in section \ref{sec:distancedriven}, as the trip meter will not update the moving time or distance driven if the GPS signal is lost. The moving time is displayed below the elasped time and above the distance driven in Figure \ref{trip}

\subsection{Average Speed}

When reviewing the trip meter data, it is useful to know the average speed the car was traveling during the trip. Having this information allows the driver to better understand the characteristics of the drive. This element is also easy to calculate, as the time elapsed and the distance driven are already available. Equation \ref{eq:averagespeed} shows the formula used to calculate the average speed. The calculated value is displayed to the right of the elapsed time in figure \ref{trip}.

\begin{align}
\label{eq:averagespeed}
\mathrm{Average Speed} &=\frac{\mathrm{Distance Driven}}{\mathrm{Time Elapsed}}
\end{align}

This value is calculated whenever the distance driven or elapsed time values are updated. As this value is using two calculated values, it does not need to worry about discrepancies in time or the loss of the GPS signal.

\subsection{Average Moving Speed}

The average moving speed is like the average speed. The only difference is that it uses the moving time to calculate the speed, rather than the elapsed time. This is done using the same equation \ref{eq:averagespeed}, only substituting "Time Elapsed" for "Moving Time". This value provides the average speed of the car when it was actually being driven, thus ignoring time spent waiting in traffic.

\subsection{Reset}

The final functionality of each independent trip meter is the reset button. This button resets the trip meter it is attached to. The distance driven, moving and elapsed time counters will all display zero, and the average speed calculators will display zero. As each trip meter is independent, one can be reset without affecting the other. To reset the trip meters, the driver just has to press the reset button, located below the average moving time and to the right of the distance driven in Figure \ref{trip}.

\subsection{Current Speed}

The trip meters provide statistics on where the car has been driven, but do not provide much insight into the instantaneous speed of the vehicle. As this variable is already being used in calculations it is trivial to display it to the driver. This is displayed using a simple digit element, and appears in the top right corner of Figure \ref{trip}.

\subsection{Time Trial Data}

A common metric in measuring the performance of cars is to measure how long the car takes to achieve a certain speed. Usually this requires expensive equipment in order to accurately measure the time and speed data. As the information exists inside the trip meter in some form already, it is useful to display a less accurate version of this time trial data. By recording the time it takes to reach 50 or 100 km/h the operator is able to have quick feedback on the performance, without having to setup lots of equipment. The flow chart of this calculation is given by Figure \ref{timetrialflowchart}. The results of this are displayed below the current speed in Figure \ref{trip}

\figuremacro{timetrialflowchart}{Time Trial Data flow chart}{The program flow used to calculate the 0-50 km/h and 0-100 km/h time trial data}

An important condition on this flow chart is that the zero time must be set before any calculations are performed. The zero time is the time at which the car was last traveling 0~km/h. This time will be reset whenever the car is stopped, so the calculation requires no input from the user. Also present in the flow diagram is that the system will display the best time recorded. If there is a previous best time, and a new one is achieved, the new one will automatically be displayed. This allows the user to ignore the trip meter panel entirely, and be able to trigger and record some performance data on normal drives.

\subsection{Persistance}

TODO: talk about file format and persistance


% ---------------------------------------------------------------------------
%: ----------------------- end of thesis sub-document ------------------------
% ---------------------------------------------------------------------------


%
%: ----------------------- contents from here ------------------------

\section{Arduino Digital Input Module}
\label{sec:arduino}

The car has access to many physical inputs that would be useful to monitor and record. Variables such as the state of the air-conditioning or the radio are useful aspects to monitor. These are currently exposed via bare wires inside the vehicle. As these signals are simple digital logic, they need a way to interface with the controller in order to be used in the system.

Previously the in-built FPGA on the Eye-bot was used to accomplish this end \cite{thesis_varma}. This method requires complex cabling to the inside of the Eye-bot. Use of the FPGA also requires special code to be written to interface with the FPGA, which will be different for different kinds of FPGA devices. This makes maintaining this design difficult, and ties the code in with the specific Eye-bot it was developed for.

Investigation was done into a top16 digital input and output module. This module has 8 digital input lines, and 8 digital output lines. This allows for a total of 8 inputs to be read. While this would be enough to satisfy the input variables, it suffers from some drawbacks. This board was found to use an FTDI chip for usb serial communication. This chip appears as a serial port when the correct drivers are present. Sadly the FTDI drivers for the older Linux kernel were not stable, and it was not possible to use this board in the system.

As the current methods were unsuitable to fulfill this role, a new digital interface system was developed in order to satisfy the requirements needed. This new board would have digital inputs, analogue voltage inputs, and be able to read other signals generated by the car.

\subsection{Speedometer and Tachometer}

A desired feature would be to record more complicated signals from the car, such as the current speed. While the GPS can provide speed readings, a more accurate source of speed data is available. This source is the cars built-in speedometer. The speedometer and tachometer use hall effect sensors in order to read the rotational rate of the gearbox and motor respectively. 

\figuremacro{speedvsfrequency}{Speed (km/h) vs Frequency (Hz)}{}
\figuremacro{rpmvsfrequency}{RPM vs Frequency (Hz)}{}

Figures \ref{speedvsfrequency} and \ref{rpmvsfrequency} show the output on the in-built dash in response to various pulse train frequencies. Both these graphs show that the relationship between the frequency and desired variable are highly linear. Thus it is possible to count the amount of pulses that have transpired and perform a calculation in order to determine the cars current speed.

\subsection{Hardware}

The hardware used to accomplish this task is an Arduino Uno compatible board. This board provides 14 Digital IO pins and 6 Analogue Input pins \cite{arduinospecs}. The Atmega chip inside the board also has inbuilt timers, which can be used to implement the frequency component of the requirements. The board is quite small in size, and only needs 5v to run. Figure \ref{ArduinoUno} shows the Arduino Uno board.

\figuremacroW{ArduinoUno}{Arduino Uno}{}{0.5}

\subsection{ Drivers}

One advantage of the Arduino Uno board over other Arduino boards is that it implements the cdc-acm device drivers that were used in section \ref{sec:gpsdrivers}. The fixes to the drivers implemented previously allow the Arduino Uno to work with the system without any hassle. 

\subsection {Design}

\subsubsection{Arduino}

The program designed for the Arduino uses the internal interrupts of the Arduino in order to keep track of time and the amount of pulse that have occurred. The Arduino contains an ATmega328 \cite{arduinospecs}. This chip contains internal circuitry that is able to count the number of pulses independently of the current system clock. There are two pins available on the micro-controller to facilitate this functionality, and two frequencies the system is interested in recording.

\begin{table}
\begin{center}
    \begin{tabular}{|l|l|}
        \hline
        Property & Value \\ \hline
        Baud Rate    & 115200 \\
        Data Bits    & 8    \\ 
        Stop Bits    & 1    \\ 
        Parity       & None \\ 
        Flow Control & None \\
        \hline
    \end{tabular}
	\caption{Connection settings for Arduino}
\end{center}
\end{table}

The main logic of the Arduino program is fired in an interrupt that fires at a frequency of 125Hz. This interrupt checks the overflow of the 8 bit counter, and stores the result so it can be used later on in the calculation. This allows the program to work with larger numbers than a 8 bit byte can contain. Every 125 cycles of this interrupt, or every one second, the system calculates the number of pulses that have occurred since the last second, and stores it in a different memory location. The program also sets a flag inside the micro-controller, signaling that the data is ready to be sent via a serial link.

When the serial link is signaled, it will read all the digital inputs and analogue inputs as well as the calculated pulses, and transmit them to the device on the other end. As this is a serial stream, it must have some values that are reserved in order to synchronize the data stream. The same method as used in section \ref{sec:expertprotocol}. This uses the same header byte (0xFF) and allows for 7 bits of data to be transmitted per byte. Rather than sending multiple messages, the Arduino sends everything in one message, that is transmitted at 1Hz.

\begin{table}
    \begin{tabular}{*{7}{|l}|}
        \hline
        Byte 1 & Byte 2 & Byte 3 & Byte 4 & Byte 5 & Byte 6  & Byte 7 \\ \hline \hline
        Header & Count & 7 Bits DIO & 5 bits DIO & Analog0 MSB  & Analog0 LSB & Analog 1 MSB  \\ 
        \hline
    \end{tabular}
 \begin{tabular}{*{8}{|l}|}
        \hline
        Byte 8 & Byte 9 & Byte 10 & Byte 11 & Byte 12 & Byte 13 & Byte 14 & Byte 15\\ \hline \hline
       Analog 1 LSB & \multicolumn{4}{|c|}{Frequency 0} & \multicolumn{3}{|c|}{Frequency 1} \\
        \hline
    \end{tabular}
\begin{tabular}{*{8}{|l}|}
        \hline
         Byte 16 & Byte 17 & Byte 18 & Byte 19 & Byte 20 & Byte 21 & Byte 22 \\ \hline \hline
        \multicolumn{1}{|c|}{Frequency 1} & \multicolumn{6}{|c|}{Reserved} & Trailer \\
        \hline
    \end{tabular}
	\caption{Network protocol for the distributed system}
	\label{tab:networkprotocol}
\end{table}

\subsubsection{Host side}

On the host side, the program functions similarly to that of the GPS program, such as in Figure \ref{gpsflowchart}. This module of course, will not set the system date, but otherwise they perform similarly, attempting to read as much data as possible into a buffer, and then finding the correct section to synchronize on.







% ---------------------------------------------------------------------------
%: ----------------------- end of thesis sub-document ------------------------
% ---------------------------------------------------------------------------


%
%: ----------------------- contents from here ------------------------

\section{Cost Panel}

\figuremacro{money}{Savings Panel}{}


% ---------------------------------------------------------------------------
%: ----------------------- end of thesis sub-document ------------------------
% ---------------------------------------------------------------------------


%
%: ----------------------- contents from here ------------------------

\section{About}

% ---------------------------------------------------------------------------
%: ----------------------- end of thesis sub-document ------------------------
% ---------------------------------------------------------------------------




%: ----------------------- contents from here ------------------------

\section{Redraw Performance}

Due to the limited resources of the system, care must be taken when attempting to update the screen. As as windowing system was already being developed, it was desirable to have this system absract these actions away from the programmer. By using the toolkit the programmer can trigger when an element should be draw, and specify the drawing code to draw the element. The absraction means that the developed code will never be called directly by the programmer, it is all taken care of by the underlying mechanisms.

\subsection{Refresh on Arrival}

Whenever a new message is recieved by the user interface, it would seem appropriate to process that message instantaneously. Figure \ref{screenaccess} shows the flow of this methodology. A message is recieved, then the system processes the message, redraws the screen, and then starts over again. While this is a working solution, it does present problems in regards to performance and the overall usability of the system.

\figuremacro{screenaccess}{Processing message flow chart}{This figure shows the naive approach to processing and displaying recieved message data}

The performance penalty in such a design is not immediately obvious. Data that is recieved should be displayed instantanouelsy. Inspecting Figure \ref{screenaccess} further does help highlight the problem that occurs in this situation. While the system is processing or displaying the message, it is unable to process anymore messages. This can be a problem when the source of the messages is generating messages faster than the device can process.

\subsubsection{Message speed greater than redraw rate}

In situations such as the BMS module, the program was able to sufficiently cope with the input. This lead to all the messages being instantly removed from the network layer when they arrrived. Thus the program was always in the state of "waiting for a message". However another important sensor caused problems with this method. This sensor was the one designed to read the GPS signals. Messages containing GPS information where generated at a rate of 10 messages per second, or 10hz. This speed led to the situation where when a new message arrived, the process was still in the "Draw" state. This would cause the message to be delayed on the network layer.

 The rate of input regarding the GPS module was constant, these messages would continually build up on the network layer. Any new messages that were transmitted would be dropped when they were attempted to be sent. While this is acceptable, eventually space would be cleared for new messages, it would lead to alot of new messages to be dropped. The other issue that occured here, was the new messages appeared at the back of the "Queue". Until all the older messages were processed, the new ones would not be seen. This is expected from how ZeroMQ functions \cite:{zeroMQ\_internals}. However, this meant that there was a significant delay until new data was seen.

The delay that occured was proportional to the size of the ZeroMQ Message Queue \cite:{zeroMQ\_internals}. This delay could be up to a couple of seconds, which is unacceptable for live feedback to the driver. Reducing the size of the Message Queue helped alleviate the problem, however the screen would still attempt to redraw as fast as the messages where recieved. This also led to the interface portion of the system hogging the CPU. Thus an alternate method of dealing with screen updates was developed.

\subsection{Add to Queue}

\figuremacro{screenaccessnew}{Appending to Queue}{This figure shows the flow of appending screen refreshes to a queue}


\subsection{Redrawing the Screen}

The previous section discussed the problems that occured with allowing the screen to update whenever a message was recieved. The biggest performance penalty that occured was not from the actual processing of the data, but from having to display it on the screen. All of the processing is relatively trivial computation wise. It is the transfer of variables into various memory locations so they can be accessed when the screen is displayed. Copying a variable itself is trivial, however processing that variable for display on screen is incredibly intense.

\subsection{Redraw rate}

Inspection of the EyeLin library source code showed that the library was using a simple framebuffer in order to interact with the screen \cite:{frame\_buffer}. This framebuffer method stored the entire screen state as an 24bit image. By default, whenever one pixel was changed in this image, all the data was copied to the framebuffer again. This was incredibly wasteful, and contributed to the large delays in redrawing the screen. Many items on the screen need to be redrawn together, for instance, a digit display has three or more digits that may change from it's last appearance. By default the library would redraw three times if the number changed by a large amount. This caused alot of performance issues in previous projects using these libraries see 

\subsection{Batch Redraw}

In order to offset this problem the way in which the queue processed draws was modified. The thread that processed the queue would attempting to dequeue as much items as possible and process them as a batch. This gives more performance than attempting to redraw the screen for every change.

\subsubsection{Maximum batch size}

Attempting to remove as many items as possible is a problem in a multi-threaded system. Consider the case where a thread is adding elements, the producer thread, at the same time the drawing thread is removing them, the consumer thread. If the producer thread is operating faster than the consumer thread, the consumer thread will always be removing elements from the queue. Thus the consumer thread has a maximum amount of elements it will process at a time. This garuntees that the consumer thread will always refresh the screen. 

\subsubsection{Small batch size}

Another issue that can occur with processing drawing events in a batch format is that no new draw events may be generated. Consider that the thread is attempting to remove a certain number of redraw events, however there may only be half present inside the queue. If , for whatever reason, no more redraw events are added, the queue will wait forever attempting to remove them. This is undesirable, as some transitions may only trigger a few redraw events and do nothing more. A good example of this is a static page, like the sponsor page. It only has a few elements, the buttons and the sponsor logos, and is not dynamically updated in response to any data. If the thread was waiting for more draw events, they would never be recieved. The solution to this problem is to continue on with the drawing actions if the queue is ever empty. This prevents the thread for waiting for more events, and helps garuntee the constant refreshing of the screen.


%\figuremacro{screenflowchart}{Screen drawing flow chart}{This figure shows the managing and drawing of elements inside the drawing queue}






% ---------------------------------------------------------------------------
%: ----------------------- end of thesis sub-document ------------------------
% ---------------------------------------------------------------------------






% ---------------------------------------------------------------------------
%: ----------------------- end of thesis sub-document ------------------------
% ---------------------------------------------------------------------------

