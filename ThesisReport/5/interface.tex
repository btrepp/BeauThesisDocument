% this file is called up by thesis.tex
% content in this file will be fed into the main document

%: ----------------------- name of chapter  -------------------------
\chapter{Windowing Toolkit} % top level followed by section, subsection


%: ----------------------- paths to graphics ------------------------

% change according to folder and file names
\ifpdf
    \graphicspath{{5/figures/PNG/}{5/figures/PDF/}{5/figures/}}
\else
    \graphicspath{{5/figures/EPS/}{5/figures/}}
\fi

%: ----------------------- contents from here ------------------------

\section{Motivation}

In order to improve stability of the system, a windowing toolkit was developed to display information to the user. Interactions with the user occur in only a few pre-definable ways. The user will either view data that the system has produced, or press buttons on the screen in order transition the display to another screen. The currently developed software (EyeLin) provided very low level access to complete these actions. This allows for greater flexibility in developing using the software, but makes it  more confusing to deal with, and contributes to bugs in the software. In order to alleviate this problem, a high-level abstraction was developed on top of the existing software. This abstraction allows the low level functionality, such as interacting with the touchscreen, to be hidden. This simplifies development when layout the user interface. This also allows for code to be written and debugged once, for example the toolkit completely removes interactions with the touchscreen, allowing buttons to be added in a much simpler fashion.

As other requirements of the system required C++ libraries, the windowing toolkit was implemented using C++. This is an object oriented language that provides a few distinct advantages over the C language that is traditionally used on embedded devices such as this. The first advantage is the use of object orientation. When constructing a user interface, it is much easier to understand the different elements of the interface as unique objects. By thinking of a text display as a single object, it becomes more intuitive to manipulate. The other advantage in the use of C++ is that it allows for objects to inherit from other objects. This use of polymorphism allows elements to implement the functionality they require by inheriting from specifications in the toolkit. The toolkit is able to call these new functions due to use of virtual methods.

The toolkit has been written in a way that the programmer utilizing it does not need to understand anything about threads. While the mechanics inside the toolkit do use threads, the interaction with this threads is completely abstracted away. It is not possible to directly manipulate the underlying threads with the toolkit. This removes any problems with synchronization between threads, as they cannot be manipulated. There are three threads running inside the toolkit. One thread is responsible for manipulating the screen, one thread is responsible for responding too events occurring on the touchscreen  and the last thread is responsible for reading and processing the network messages.

The final advantage of developing this toolkit, is that it is not limited to use in this project. As the toolkit is written in a generic way, with the exception of being able to receive the messages transmitted over the network, it is able to be deployed on future projects. This allows future projects to have a rich user interface, while keeping the high-level abstraction in place.

\section{Elements}

The windowing toolkit consists of pre-made classes that the are either used directly, such as the digit display element, or are inherited from, such as the base or runnable classes. Figure \ref{uiframework3} shows the classes developed in the toolkit.

\figuremacro{uiframework3}{UML diagram of the window toolkit}{}

\subsection{UIElement}

The UIElement is the most basic class definition in the toolkit. It is an abstract class that can never be instantiated. Its purpose is to define methods for interacting with the screen and other objects. It also provides default functionality for most methods, this allows all the classes that inherit from it to function the same way.

\subsubsection{addChild()}

An important definition of the UIElement class, is that it may contain any number of other UIElement classes inside it. To add another UIElement class the method addChild() is called. This stores the child element inside a C++ vector whose length is only limited by the amount of ram inside the machine. The advantage of this is that actions can be performed on a UIElement and all it's children. If a element needs to be drawn, all it's children will be drawn too or if an element is disabled, all it's children will be disabled too.

\subsubsection{draw()}

 The most important method of the UIElement class is the draw method.  This method definition is not implemented in the UIElement class, it is defined a virtual abstract method. All classes that can be instantiated must implement this function call. The purpose of this function call is to allow the user to specify the low level commands that are used to display this element. This can include drawing lines, squares, or setting individual pixels. This method should be implemented by the programmer, but should only ever be called by the mechanics of the toolkit. The developed of the user interface should never directly call this method.

\subsubsection{enqueueDraw()}

enqueueDraw() is called whenever the system, or the programmer, wants to trigger a refresh of the screen. This will signal the toolkit that a redraw should be prepared. The purpose of this method is two-fold. Firstly it allows the toolkit to perform optimizations of the draw function in order to maximize speed (see \ref{Redrawing the Screen}). Secondly it's implementation has O(1) complexity. This means that the call to enqueueDraw() completes in constant time. This is used for performance reasons, as whatever thread has called enqueueDraw(), will not need to wait for the screen to be redrawed, it will return instantly. The draw will be scheduled to occur some time after enqueueDraw() is called. By default this method will also call the enqueueDraw() method of all it's children, allowing entire sections of the display to be redrawn using one function call.

\subsubsection{animate()}
TODO

\subsubsection{setActive()}

A property that is required of any draw-able object inside the toolkit is whether it is current being displayed to the user. This property allows elements to exist in the machines memory, but only be draw if they are current being displayed. To manipulate this property, the method setActive() is called. This method allows the state to be set to either true or false, meaning that the object will be drawn or not drawn respectively. If the active state is false, calls to enqueueDraw() will be processed, but the call to the draw() function will be skipped. Thus individual elements of the display can be hidden at will. This method will also call the setActive() method of all the children of this element. Thus allowing sections of the display to be hidden with one function call.

\subsubsection{isActive()}

This method will return the current state of the element. This is used to check whether the current element is being drawn or not. This method is called internally by the screen drawing mechanics. It can also be used in order to check whether the element is being displayed, and perform different tasks if it is not being displayed.

\subsection{base}

The base element represents the panels or windows that are being displayed to the user. An important property of the base element is that it is defined to occupy the whole screen. This element will draw the entire width of the screen, which will clear any old draws that may still be present. Another extended property of the base element is that it maintains a list of all the other base elements that are present. This is used in order to allow for global navigation buttons. Rather than layout buttons in the same location on every screen, a button can be added as a global button. This will ensure that it appears on all the screens present in the list. This allows the buttons and their location on the screen to be defined once, making the final program more stable and simpler to understand.
For further discussion on the button element see \ref{Button}.
The base class itself is abstract, it cannot be instansiated. There is no way to display a "default" base element. In order to build a panel, the panel must inherit from the base element, and implement at minimum the abstract function getButton().

\subsection{getButton()}

In order to transition to a panel, an action must be undertaken by either the user or the system. The most common way of transisitioning would be when the user wants to display a different screen. Typically this would occur by the use of pressing buttons. This is why any base panel must implement the getButton() function. This function returns a button object that contains the image data to display for this button, and the action to undertake when the button is pressed. This action will typically be a call to the activate() function of the panel, though other actions can be called before the call to activate().

\section{Subscriber Queue}

As the interface will receive many different types of messages that are described in section \ref{sec:design}, the ability to easily process these messages was added to the framework. Three classes were developed in order to facilitate this functionality. A UML diagram of the classes is shown in figure \ref{uiframework2}. This functionality uses a listener based system, in order to direct each message to the class that needs the information.

\figuremacro{uiframework2}{UML Diagram of messaging system for the window toolkit}{This figure shows the UML diagram of the message subscriber, filter and receiver classes}

\subsection{Subscriber}

The subscriber class is the class that takes care of reading the messages from the network. This class maintains a instance of a ZeroMQ socket, and registers to receive all messages that are sent to that socket. It is able to listen to multiple endpoints, so it can receive messages from multiple unique devices and deal with them from the one class.

The class can be run directly, or can spawn it's own thread. When the class is run, it will block indefinitely, as to direct messages to their intended destination. Thus it is recommended that the threaded functionality be utilized when this class is instantiated. 

An infinite number of listeners can be registered to this class. Each listener will specify a filter that indicates what messages it is interested in. This associates the added listener with the desired filter. If the filter does not currently exist, it will be created by the subscriber class.

When a message is received, all the implemented filters are compared to the message. If a filter matches the received message, the message is forwarded to all the classes that are listening to this filter.

\subsection{Filter}
\label{sec:toolkitfilter}

The filter class is responsible for identifying what messages will pass this filter, and maintaining a list of classes that should receive the message if it passes. Any class implementing the listener interface can be registered to a filter.

A filter also has some other options that are useful to it's operation. It is possible to set a timeout for each filter that is created. If a message that matches the filter is not received in the length it takes the timeout to occur, the listeners of this filter will be notified. This allows the listeners to respond to lapses in communication in whatever way they want. 

A filter can also throttle the message rate. It is possible to specify a minimum time between messages, in order to reduce processing time used by the CPU. If the message rate is greater than the throttle rate, messages will be discarded at the filter level. This can be used to help prevent too many messages from appearing at a listener that cannot cope.

\subsection{Listener}

A listener is a abstract class that defines the function that will be called when a valid message arrives. If a class implements the recieveEvent method, whenever a message that it is interested in arrives it will call this method. This arguments passed through are the length of an array of bytes, and a pointer to the start of this array. This allows the class the receive the contents of the message, and use it in whatever way it pleases.

\section{Button Translation}
\label{sec:button_thread}

Discussed in section \ref{sec:button} is the class that represents buttons on the display. In order to interact with these buttons, and not interfere the operation of the other aspects of the system, a separate thread must be utilized. This thread has the sole responsibility of navigating through the display that is shown to the user, and triggering the actions of the button. The functionality of this thread is implemented in the activate() function of the base class. This ensures that any element created that will contain buttons has the ability to trigger them.

The existing libraries for interfacing with the touch screen written by Sommer provided an integer indicating which button had been pressed \cite{thesis_sommer}. This method, while usable, is complicated and can cause many problems when determining what action had occurred. In order to simplify this, the activate() loop will take this integer value, and determine which button element it corresponded to. 

The buttons on any element are stored in a vector that is separated from other elements. This provides a list of the buttons that have been added to the device in order. When refreshtouchmap() is called on an element, the loop will setup the touch regions using methods provided by Sommer's library. Sommer's libray will associate each button to a bit in a 32 bit register. Each bit corresponds to a section of the screen for an individual button. This allows multiple buttons to be pressed at the same time.

The relationship between the value in the register and the position in the vector is given by equation \ref{eq:buttonregister}. Thus it is possible to convert the value returned from Sommer's library into an index in the vector. Knowing this, the loop will retrieve the runnable action from the button at this vector location. If the runnable action is valid, the result will be returned to the framework. When the framework recieves a runnable action, it will run it. As a safe-guard, if a runnable action is invalid for whatever reason, it will perform the action to display a default panel. As all calls to activate will return to the highest level of the framework, there is no nesting of panels in the system memory. Hence activate will always return to the top level.

\begin{align}
\label{eq:buttonregister}
\mathrm{Button}_i &= \log2{(\mathrm{Register}) }
\end{align}



%
%: ----------------------- contents from here ------------------------

\section{Battery}

The battery panel is the main display panel used in the user interface. It is the first panel displayed to the user when the system turns on.  It displays five important pieces of information to the user operating the vehicle. This panel listens to both the battery (TBS) and gps messages. It requires the battery messages to display the battery voltage, current and charge to the user. These are displayed using the digitelements mentioned in \ref{sec:digitelement}. It also uses a custom charge element class that exists outside of the framework to draw a battery on the center of the screen. This green bar on this battery will decrease in proportion to the charge remaining in the car, providing a quick visual indicator for how much charge is still stored. 

This panel also displays the speed in km/h, which is why it has to register to the gps module. The speed is included so the user does not have to interact with the screen while driving. This panels main purpose was to provide a quick overview to data points that are immediately of concern to the driver. 

Displayed in the top left corner is a rudimentary calculation of how much distance is remaining in the car. Range tests conducted by the REV team indicated that this distance was 80km from a full charge. To provide some security while driving, the interface assumes that the max distance the car will travel on a full charge is 70km. It uses this to calculate the distance remaining according to the formula \ref{eq:distanceremaining}

\begin{equation}
\label{eq:distanceremaining}
70=10
\end{equation}

\figuremacro{batt}{Battery state panel}{Screenshot of the battery state panel, showing the voltage, current, charge, speed (via gps) and distance remaining}



% ---------------------------------------------------------------------------
%: ----------------------- end of thesis sub-document ------------------------
% ---------------------------------------------------------------------------


%
%: ----------------------- contents from here ------------------------

\section{Maps}

A common, yet useful driver aid is displaying the map of the current location. This is what the maps panel does. It listens to the GPS messages to determine the position of the car. A screen-shot of this panel is shown in Figure \ref{map}. This panel uses the full area to display the current location of the vehicle on the screen. The actual position of the vehicle is centered on the middle of the screen, allowing the driver to view streets and landmarks relative to his current position. The panel features several levels of zoom, controlled by the bar on the top right corner on the screen. The plus button will zoom in, and move the slider to the right, while the minus button will zoom out, and move the slider to the left. The system supports various levels of zoom, being able to display a few buildings relative to the car or being able to display the surrounding suburbs.

\figuremacro{map}{Map display panel}{Screen-shot of the map display panel, showing the map of the current location and the map controls visible}

\subsection{Map Data}

In order to display the map images, the map data must first be obtained and stored. There are many possible methods for doing this, ranging from creating the maps as needed, or downloading them from external services and storing them in a cache. The relatively low CPU power of the eye-bot m6 makes rendering the maps on-the-fly a undesirable prospect. Open source map data for the entire planet results in a file that is 18GB in size \cite{planet_osm}. This file is much too large to store locally, and this file is actually storing a compressed version of the data. Even if it were possible for the eye-bot to store this file, via the use of external storage, it would be a large strain on the CPU to convert the street level data into viewable maps. It would also require many other pieces of software to be installed on the device, making it much more complicated to manage.

Another possibility is to use an existing Internet based map server and download the map imagery as needed. The has several downsides. Foremost it requires an Internet connection whenever to display the maps. The system does have a 3G connection installed, but this cannot be considered a dependable communication channel. It is highly likely to drop out, and is limited in coverage to the areas in which it has reception. Even without these issues, most map-servers do not allow you to download maps in bulk, as this violates their usage policies \cite{tile_usage_policy}. This makes this method undesirable as it is not suit-able for downloading maps in bulk, or as needed. Attempting to pre-download all these maps using a non-3g link would result in a violation of the usage policy.

The method chosen to obtain the map data was to pre-create create the map data using a more powerful machine. A map-server was setup and loaded with all the street data for the oceania region. For more information on the map-server setup see appendix \ref{app:mapserver}. This method overcomes the problems of the previously mentioned methods. All the processing is done on the much faster machine in advance. The area processed in advance is defined by the properties in table \ref{tab:mapbbox}. This area is depicted by the image shown in Figure \ref{mapperth}. The expanse of these maps covers all of metropolitan Perth. In future more maps can easily be processed, however this will result in more storage space being required. The current settings are a good balance between storage and expanse of data. This is because the map data will be less useful outside of the city, and the car is typically not driven any further than the pre-rendered maps. The current space usage of the map data is given in table \ref{tab:mapdata}. As this is much much larger than the internal 16MB of flash storage the eyebot has, it must be stored on external storage. Luckily, storage is now cheap, and 8GB usb thumb drive has enough performance and space to store and serve the image data.

\begin{table}[htdp]
\begin{center}
	
   \begin{tabular}{|l|l|}
        \hline
        Property           & Value           \\ \hline
        Minimum Zoom Level & 11              \\ 
        Maximum Zoom Level & 18              \\ 
        Top Left           & 115.687,-31.71  \\ 
        Bottom Right       & 116.508,-32.253 \\
        \hline
    \end{tabular}
	\caption[Properties of the pre-rendered map data]{The properties of the pre-rendered map data}
	\label{tab:mapbbox} 
\end{center}
\end{table}

\figuremacro{mapperth}{Pre-rendered map size}{This figure shows the area that has been pre-rendered for use in the map panel display}

\begin{table}[htdp]
\begin{center}
    \begin{tabular}{|l|l|}
        \hline
        Property              & Value                             \\ \hline
        Zoom Levels           & 7                                 \\ 
        Number of Files       & 435,456 \\ 
        Total Size            & 580.2 MB                          \\
        File Resolution            & 256*256 pixels                          \\  
        File Format            & Palette PNG                         \\ 
        Time taken to process & Approximately 1 Day                          \\
        \hline
    \end{tabular}
	\caption[File statistics of map data]{File statistics of map data}
	\label{tab:mapdata}
\end{center}
\end{table}

\subsection{Tiling}

As mentioned previously, the map data is much larger than the internal storage of the eye-bot. It is also much much larger than the operating systems 64MB of ram. This makes it impossible for the entire map data to be loaded inside any program to be displayed to the user. In order to overcome this limitation, the map panel only loads a small subset of the map data at a time, and displays it using a method called tiling. Rather than have one giant map that displays all of the information, and sliding this around to view the correct part, this method divides the map into many smaller maps. These smaller maps are referred to as tiles. Each tile consists of a 256 pixels wide by 256 pixels high image. These images are combined in order to display the current location, as seen in figure \ref{tiles}. By loading the adjacent tiles in all directions the screen will always have data to display.

\figuremacro{tiles}{Tiling Maps}{This figure shows the tiles that are loaded, labeled 0-8, and the area of the screen that is able to view them.}

\subsubsection{Converting GPS Co-ordinates}

The GPS outputs the current position in latitude/longitude format. This format must be converted so that the correct tiles may be loaded. To do this the GPS co-ordinates are converted to grid based co-ordinates using a mercator projection \cite{slippy_map_tilenames}. The equations to convert to this format are given by \ref{eq:gpstotiles}. It is possible to convert from a grid co-ordinate back to a GPS co-ordinate using the equations given by \ref{eq:tiletogps}. This of course will only be able to return the top left corner of the tile in GPS co-ordinates, and not the original position.

\begin{equation}
\label{eq:gpstotiles}
70=10
\end{equation}
\begin{equation}
\label{eq:tilestogps}
70=10
\end{equation}

With the GPS co-ordinate converted it is possible to locate the tile to display to the user. Rather than utilize lookup files to locate the tile, a specific folder structure is used to instantaneously load the file. This is done by storing the files in the format \emph{/zoom/tilex/tiley.png}. This allows the system to directly open the file and display it. Thus even if the entire world was loaded in the maps folder, for many different zoom levels, the display time of the current location would not change.

% ---------------------------------------------------------------------------
%: ----------------------- end of thesis sub-document ------------------------
% ---------------------------------------------------------------------------


%
%: ----------------------- contents from here ------------------------

\section{Trip Meter}

A useful driver aid that is common on vehicles is that of a trip meter. Traditionally this component records the distance the car has traveled since the trip meter was set. This functionality is usually a result of the simple systems in place, and is tied to the revolutions of the wheels on the vehicle. As this system has more hardware at it's disposal, the trip meter can implement more functionality than a standard trip meter, making it much more useful in examining the performance of the car. Figure \ref{trip} shows the trip meter panel being displayed on the screen.

\figuremacro{trip}{Trip Meter Panel}{This figure shows two independent trip meters and the best record speed data}

The first unique point of this trip meter, is that it contains two independent meters. This is useful as it allows the driver to evaluate the statistics of two overlapping trips. One trip meter can be used to record the distance traveled since the car was last charged, while the other can be used to record the distance traveled in the last week or month. This independence allows the operator to decide how best to use the trip meter data, resulting in a high level of flexibility. In figure \ref{trip} the trip meters are located on the left, sitting above each other.

Each trip meter records the distance traveled, the time elapsed since the meter was started, the time the car has been moving since the meter was started and calculations based on the elapsed and moving time. These statistics are displayed live to the user, but are not logged, as the logging functionality  is taken car of by a different component in the system. The trip meter panel also displays the current moving speed in the top right. Below the moving speed are the best run records in seconds. This allows the driver to have quick feedback as to how the car is performing, without having to do lots of processing on logged data.

\subsection{Distance Driven}
\label{sec:distancedriven}
The most important part of a trip meter, is the distance that the meter has recorded. This is shown on figure \ref{trip} at the bottom of each trip meter. The meter stores the distance driven internally as a double length floating point number, but displays it on the screen as a rounded integer. This is done in order improve precision for later calculations as other values will depend on the distance that has been driven. Equation \ref{eq:distancedriven} shows the the distance is calculated based on the GPS position.

\begin{align}
\label{eq:distancedriven}
\mathrm{Distance} &=\mathrm{Distance}_\mathrm{lastrun} + \mathrm{Speed}(\mathrm{Time}_\mathrm{now}-\mathrm{Time}_\mathrm{lastrun})
\end{align}

The method for working out is a continuous function that is based on the last known distance the car has traveled. This method was chosen as it does not require any information other than the last time the formula was run, and the last distance calculated. This also makes the trip meter flexible in that the distance calculation does not need to be processed at exact intervals. If messages are dropped for whatever reason, the calculation will still take place, though it will not be as accurate as it could be. The calculation will be able to cope with fluctuations in the message timing, and can adapt to the speed of the GPS being used.

The downside of this method, is that big changes in time can cause problems with the calculation. If the signal drops out for an extended period of time, such as going through a tunnel, the calculation in \ref{eq:distancedriven} would have a big margin for error. In order to prevent this, the trip meter will ignore large time differences. If two calculations are over 10 seconds apart, the result will not be trusted, and not be used in the calculation. This prevents GPS signal loss from having an adverse effect on the trip meter calculations, but does impose a limit on the trip meter.

The limitation of this method of calculating the distance driven is that it relies on the GPS messages being sent to it. If the GPS signal is lost, the distance driven will not be increased. This is a limitation imposed by the use of the GPS sensor, and cannot be avoided, as attempting to guess the distance driven while the GPS signal has been lost has a very high probability of being incorrect.

\subsection{Time Elapsed}

Another variable displayed on the trip panel is the time elapsed. This is simply the time elapsed since the trip meter was last reset. This time increments even while the GPS signal has been lost, performing a stop-watch like action on the trip. While it may seem natural to just record the time that the trip meter was started and subtract it from the current system time, this would lead to problems during the system startup. The time needs to increment even while the GPS is connecting, and must be resistant to changes in the systems internal clock. Thus the time is calculated similar to section \ref{sec:distancedriven}. The formula used to calculate the elapsed time is given in equation \ref{eq:timeelapsed}. The time elapsed is displayed as the highest element of the trip meter in Figure \ref{trip}

\begin{align}
\label{eq:timeelapsed}
\mathrm{TimeElapsed} &=\mathrm{TimeElapsed}_\mathrm{lastrun} + (\mathrm{Time}_\mathrm{now}-\mathrm{Time}_\mathrm{lastrun})
\end{align}

Much like the distance, this method of calculation depends on the last known values. This means it does not matter at the actual time the system trip started once the timer has been running. This makes it resistant to changes in the system time, and thus makes the timer more robust. 
This timer records the time elapsed on the nanosecond level, as it is used elsewhere in calculations. For display, the timer converts these values into the traditional hours, minutes, seconds format that is easy for the operator to read.

\subsection{ Moving Time}

An aspect of the cars telemetry that would be interesting to the driver is the cars moving time. This is defined as the time in which the car has spent in motion. The main use of this data point is to contrast it against the elapsed time, to highlight how long the car has spent sitting still in traffic. This variable is also useful to record for future calculations, such as working out the average speed of the trip. This element is calculated according to equation \ref{eq:timeelapsed}, except that it will not update if the current speed of the car is 0~km/h. As such this element requires the speed of the car to be processed, so it cannot be calculated when the GPS signal is lost. This fits in with the functionality defined in section \ref{sec:distancedriven}, as the trip meter will not update the moving time or distance driven if the GPS signal is lost. The moving time is displayed below the elasped time and above the distance driven in Figure \ref{trip}

\subsection{Average Speed}

When reviewing the trip meter data, it is useful to know the average speed the car was traveling during the trip. Having this information allows the driver to better understand the characteristics of the drive. This element is also easy to calculate, as the time elapsed and the distance driven are already available. Equation \ref{eq:averagespeed} shows the formula used to calculate the average speed. The calculated value is displayed to the right of the elapsed time in figure \ref{trip}.

\begin{align}
\label{eq:averagespeed}
\mathrm{Average Speed} &=\frac{\mathrm{Distance Driven}}{\mathrm{Time Elapsed}}
\end{align}

This value is calculated whenever the distance driven or elapsed time values are updated. As this value is using two calculated values, it does not need to worry about discrepancies in time or the loss of the GPS signal.

\subsection{Average Moving Speed}

The average moving speed is like the average speed. The only difference is that it uses the moving time to calculate the speed, rather than the elapsed time. This is done using the same equation \ref{eq:averagespeed}, only substituting "Time Elapsed" for "Moving Time". This value provides the average speed of the car when it was actually being driven, thus ignoring time spent waiting in traffic.

\subsection{Reset}

The final functionality of each independent trip meter is the reset button. This button resets the trip meter it is attached to. The distance driven, moving and elapsed time counters will all display zero, and the average speed calculators will display zero. As each trip meter is independent, one can be reset without affecting the other. To reset the trip meters, the driver just has to press the reset button, located below the average moving time and to the right of the distance driven in Figure \ref{trip}.

\subsection{Current Speed}

The trip meters provide statistics on where the car has been driven, but do not provide much insight into the instantaneous speed of the vehicle. As this variable is already being used in calculations it is trivial to display it to the driver. This is displayed using a simple digit element, and appears in the top right corner of Figure \ref{trip}.


% ---------------------------------------------------------------------------
%: ----------------------- end of thesis sub-document ------------------------
% ---------------------------------------------------------------------------


%
%: ----------------------- contents from here ------------------------

\section{Trip Meter}



% ---------------------------------------------------------------------------
%: ----------------------- end of thesis sub-document ------------------------
% ---------------------------------------------------------------------------


%
%: ----------------------- contents from here ------------------------

\section{Trip Meter}



% ---------------------------------------------------------------------------
%: ----------------------- end of thesis sub-document ------------------------
% ---------------------------------------------------------------------------


%
%: ----------------------- contents from here ------------------------

\section{About}

% ---------------------------------------------------------------------------
%: ----------------------- end of thesis sub-document ------------------------
% ---------------------------------------------------------------------------




%: ----------------------- contents from here ------------------------

\section{Sharing Resources}

\subsection{Refresh on Message}

Whenever a new message is recieved by the user interface, it would seem appropriate to process that message instantaneously. Figure \ref{screenaccess} shows the flow of this methodology. A message is recieved, then the system processes the message, redraws the screen, and then starts over again. While this is a working solution, it does present problems in regards to performance and the overall usability of the system.

The performance penalty in such a design is not immediately obvious. Data that is recieved should be displayed instantanouelsy. Inspecting Figure \ref{screenaccess} further does help highlight the problem that occurs in this situation. While the system is processing or displaying the message, it is unable to process anymore messages. This can be a problem when the source of the messages is generating messages faster than the device can process.

In situations such as the BMS module, the program was able to sufficiently cope with the input. This lead to all the messages being instantly removed from the network layer when they arrrived. Thus the program was always in the state of "waiting for a message". However another important sensor caused problems with this method. This sensor was the one designed to read the GPS signals. Messages containing GPS information where generated at a rate of 10 messages per second, or 10hz. This speed led to the situation where when a new message arrived, the process was still in the "Draw" state. This would cause the message to be delayed on the network layer.

 The rate of input regarding the GPS module was constant, these messages would continually build up on the network layer. Any new messages that were transmitted would be dropped when they were attempted to be sent. While this is acceptable, eventually space would be cleared for new messages, it would lead to alot of new messages to be dropped. The other issue that occured here, was the new messages appeared at the back of the "Queue". Until all the older messages were processed, the new ones would not be seen. This is expected from how ZeroMQ functions \cite:{zeroMQ\_internals}. However, this meant that there was a significant delay until new data was seen.

The delay that occured was proportional to the size of the ZeroMQ Message Queue \cite:{zeroMQ\_internals}. This delay could be up to a couple of seconds, which is unacceptable for live feedback to the driver. Reducing the size of the Message Queue helped alleviate the problem, however the screen would still attempt to redraw as fast as the messages where recieved. This also led to the interface portion of the system hogging the CPU. Thus an alternate method of dealing with screen updates was developed.


\figuremacro{screenaccess}{Processing message flow chart}{This figure shows the naive approach to processing and displaying recieved message data}

% ---------------------------------------------------------------------------
%: ----------------------- end of thesis sub-document ------------------------
% ---------------------------------------------------------------------------



\input{5/alpha}


% ---------------------------------------------------------------------------
%: ----------------------- end of thesis sub-document ------------------------
% ---------------------------------------------------------------------------

