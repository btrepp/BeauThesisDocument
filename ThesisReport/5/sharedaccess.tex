
%: ----------------------- contents from here ------------------------

\section{Sharing Resources}

\subsection{Refresh on Message}

Whenever a new message is recieved by the user interface, it would seem appropriate to process that message instantaneously. \ref{screenaccess} shows the flow of this methodology. A message is recieved, then the system processes the message, redraws the screen, and then starts over again. While this is a working solution, it does present problems in regards to performance and the overall usability of the system.

The performance penalty in such a design is not immediately obvious. Data that is recieved should be displayed instantanouelsy. Inspecting \ref{screenaccess} further does help highlight the problem that occurs in this situation. While the system is processing or displaying the message, it is unable to process anymore messages. This can be a problem when the source of the messages is generating messages faster than the device can process.


\figuremacro{screenaccess}{Processing message flow chart}{This figure shows the naive approach to processing and displaying recieved message data}

% ---------------------------------------------------------------------------
%: ----------------------- end of thesis sub-document ------------------------
% ---------------------------------------------------------------------------

