
%: ----------------------- contents from here ------------------------

\section{Screen Drawing Queue}
\label{sec:screendrawing}

Due to the limited resources of the system, care must be taken when attempting to update the screen. As as windowing system was already being developed, it was desirable to have this system abstract these actions away from the programmer. By using the toolkit the programmer can trigger when an element should be draw, and specify the drawing code to draw the element. The abstraction means that the developed code will never be called directly by the programmer, it is all taken care of by the underlying mechanisms.

\subsection{Refresh on Arrival}

Whenever a new message is received by the user interface, it would seem appropriate to process that message instantaneously. Figure \ref{screenaccess} shows the flow of this methodology. A message is received, then the system processes the message, redraws the screen, and then starts over again. While this is a working solution, it does present problems in regards to performance and the overall usability of the system.

\figuremacroW{screenaccess}{Processing message flow chart}{This figure shows the naive approach to processing and displaying received message data}{0.6}

The performance penalty in such a design is not immediately obvious. Data that is received should be displayed instantaneously. Inspecting Figure \ref{screenaccess} further does help highlight the problem that occurs in this situation. While the system is processing or displaying the message, it is unable to process anymore messages. This can be a problem when the source of the messages is generating messages faster than the device can process.

\subsubsection{Message speed greater than redraw rate}

In situations such as the BMS module, the program was able to sufficiently cope with the input. This lead to all the messages being instantly removed from the network layer when they arrived. Thus the program was always in the state of "waiting for a message". However another important sensor caused problems with this method. This sensor was the one designed to read the GPS signals. Messages containing GPS information where generated at a rate of 10 messages per second, or 10hz. This speed led to the situation where when a new message arrived, the process was still in the "Draw" state. This would cause the message to be delayed on the network layer.

 The rate of input regarding the GPS module was constant, these messages would continually build up on the network layer. Any new messages that were transmitted would be dropped when they were attempted to be sent. While this is acceptable, eventually space would be cleared for new messages, it would lead to alot of new messages to be dropped. The other issue that occurred here, was the new messages appeared at the back of the "Queue". Until all the older messages were processed, the new ones would not be seen. This is expected from how ZeroMQ functions \cite{zeromq_guide}. However, this meant that there was a significant delay until new data was seen.

The delay that occurred was proportional to the size of the ZeroMQ Message Queue \cite{zeromq_guide}. This delay could be up to a couple of seconds, which is unacceptable for live feedback to the driver. Reducing the size of the Message Queue helped alleviate the problem, however the screen would still attempt to redraw as fast as the messages where received. This also led to the interface portion of the system hogging the CPU. Thus an alternate method of dealing with screen updates was developed.

\subsection{Add to Queue}

\figuremacro{screenaccessnew}{Appending to Queue}{This figure shows the flow of appending screen refreshes to a queue}


\subsection{Redrawing the Screen}

The previous section discussed the problems that occurred with allowing the screen to update whenever a message was received. The biggest performance penalty that occurred was not from the actual processing of the data, but from having to display it on the screen. All of the processing is relatively trivial computation wise. It is the transfer of variables into various memory locations so they can be accessed when the screen is displayed. Copying a variable itself is trivial, however processing that variable for display on screen is incredibly intense.

\subsection{Redraw rate}

Inspection of the EyeLin library source code showed that the library was using a simple frame-buffer in order to interact with the screen . This frame-buffer method stored the entire screen state as an 24bit image. By default, whenever one pixel was changed in this image, all the data was copied to the frame-buffer again. This was incredibly wasteful, and contributed to the large delays in redrawing the screen. Many items on the screen need to be redrawn together, for instance, a digit display has three or more digits that may change from it's last appearance. By default the library would redraw three times if the number changed by a large amount. This caused a lot of performance issues in previous projects using these libraries see 

\subsection{Batch Redraw}

In order to offset this problem the way in which the queue processed draws was modified. The thread that processed the queue would attempting to dequeue as much items as possible and process them as a batch. This gives more performance than attempting to redraw the screen for every change.

\subsubsection{Maximum batch size}

Attempting to remove as many items as possible is a problem in a multi-threaded system. Consider the case where a thread is adding elements, the producer thread, at the same time the drawing thread is removing them, the consumer thread. If the producer thread is operating faster than the consumer thread, the consumer thread will always be removing elements from the queue. Thus the consumer thread has a maximum amount of elements it will process at a time. This guarantees that the consumer thread will always refresh the screen. 

\subsubsection{Incomplete batch}

Another issue that can occur with processing drawing events in a batch format is that no new draw events may be generated. Consider that the thread is attempting to remove a certain number of redraw events, however there may only be half present inside the queue. If , for whatever reason, no more redraw events are added, the queue will wait forever attempting to remove them. This is undesirable, as some transitions may only trigger a few redraw events and do nothing more. A good example of this is a static page, like the sponsor page. It only has a few elements, the buttons and the sponsor logos, and is not dynamically updated in response to any data. If the thread was waiting for more draw events, they would never be received. The solution to this problem is to continue on with the drawing actions if the queue is ever empty. This prevents the thread for waiting for more events, and helps guarantee the constant refreshing of the screen.


\figuremacro{screenflowchart}{Screen drawing flow chart}{This figure shows the managing and drawing of elements inside the drawing queue}






% ---------------------------------------------------------------------------
%: ----------------------- end of thesis sub-document ------------------------
% ---------------------------------------------------------------------------

