
%: ----------------------- contents from here ------------------------

\section{Sharing Resources}

\subsection{Refresh on Message}

Whenever a new message is recieved by the user interface, it would seem appropriate to process that message instantaneously. Figure \ref{screenaccess} shows the flow of this methodology. A message is recieved, then the system processes the message, redraws the screen, and then starts over again. While this is a working solution, it does present problems in regards to performance and the overall usability of the system.

The performance penalty in such a design is not immediately obvious. Data that is recieved should be displayed instantanouelsy. Inspecting Figure \ref{screenaccess} further does help highlight the problem that occurs in this situation. While the system is processing or displaying the message, it is unable to process anymore messages. This can be a problem when the source of the messages is generating messages faster than the device can process.
In situations such as the BMS module, the program was able to sufficiently cope with the input. This lead to all the messages being instantly removed from the network layer when they arrrived. Thus the program was always in the state of "waiting for a message". However another important sensor caused problems with this method. This sensor was the one designed to read the GPS signals. Messages containing GPS information where generated at a rate of 10 messages per second, or 10hz. This speed led to the situation where when a new message arrived, the process was still in the "Draw" state. This would cause the message to be delayed on the network layer. As the rate of input regarding the GPS module was constant, these messages would continually build up on the network layer. 


\figuremacro{screenaccess}{Processing message flow chart}{This figure shows the naive approach to processing and displaying recieved message data}

% ---------------------------------------------------------------------------
%: ----------------------- end of thesis sub-document ------------------------
% ---------------------------------------------------------------------------

