
% Thesis Abstract -----------------------------------------------------


%\begin{abstractslong}    %uncommenting this line, gives a different abstract heading
\begin{abstracts}        %this creates the heading for the abstract page

The topic of electric vehicles is becoming increasingly popular due to rising fuel costs and growing concern over emissions. Despite this attention, most electric vehicles have little or no telemetry systems, making many aspects of there operation and efficiency a mystery. 

The aim of this project was to develop an extend-able system in order to capture various data-points that can be available in a vehicle, as well as an interface to display this data inside the car. The design developed differs from traditional embedded systems by being completely modular. It uses existing network protocols to allow the system to be distributed between various smaller embedded components. This will enable it to be easily extended, should the need for more data-points arise, and allows the use of many smaller systems to be implemented incrementally, rather than one expensive monolithic design.

In order to help facilitate robustness and code re-use, a windowing toolkit was also developed. This provides a common platform for user interaction, and defines pre-built components in order to speed up the design and implementations of the user interface.

By exposing and recording more data, deeper analysis can be done on the efficiency of the car, and help justify different technological improvements to the vehicle. The higher granularity of data acquired can also be used to analyze the economy of the vehicle in different conditions.

\end{abstracts}
%\end{abstractlongs}


% ---------------------------------------------------------------------- 
