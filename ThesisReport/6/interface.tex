% this file is called up by thesis.tex
% content in this file will be fed into the main document

%: ----------------------- name of chapter  -------------------------
\chapter{Interface} % top level followed by section, subsection


%: ----------------------- paths to graphics ------------------------

% change according to folder and file names
\ifpdf
    \graphicspath{{6/figures/PNG/}{6/figures/PDF/}{6/figures/}}
\else
    \graphicspath{{6/figures/EPS/}{6/figures/}}
\fi

%: ----------------------- contents from here ------------------------


\section{Layout}

An important aspect of user interfaces is that they must be visually impressive. This keeps the user interested in the product, and helps new users want to learn how to operate the device. The work done in the windowing toolkit allowed a much more visually impressive layout to be developed. This was achieved through the use of transparencies in order to overlay different elements on top of each other.

\subsection{Background}

As the device does not have enough power to dynamically render any sort of complex images or motions, a pre-rendered static background was used. This background was designed on another pc, and converted for use on the embedded system. It consists of two main regions. The first area is the main display region. This section is the largest of the space and exists so the current data being displayed can be laid over it. The second region is the navigation bar region. This area is at the bottom of the image, it uses a more uniform texture in order to contrast the main data display area above it. The background itself is rendered in black and white. The black and white styling allows important information to contrast against it easily. This helps highlight the information and user interface actions present to the user, and avoids a simple single-colour background on the user interface.

\subsection{Navigation Model}

Figure \ref{uinavigation} shows the basic routes throughout the user interface. This model is a generic tree, that can have as many leafs as possible. When the user wishes to navigate to a new panel, a pointer to that new panel is required. Thus as-long as the panel knows of the panel it wishes to transition to, any panel can be made active from any other panel. This allows the nested elements, such as syslog and status to navigate to any of the higher level panels easily as-well. These routes are not shown on the diagram, as there would be too many pathways to follow. If a panel returns a NULL pointer, the system will navigate to the App Panel to avoid a crash. This also provides handy functionality for traversing back to the top-most panel in the tree, as any leaf panel can request to transition to NULL and the App Panel will be displayed.


\figuremacro{uinavigation}{Basic Navigation of the interface}{The basic navigation routes through the user interface, does not show "quickbar" navigation routes}



%: ----------------------- contents from here ------------------------

\section{Overview Panel}

To facilitate easy navigation a panel was designed that shows all the different panels available. This panel is shown in figure \ref{app}. It contains 8 different aspects of the program, with room available for 12. This will allow the system to be extended in the future. 

\figuremacro{app}{Panel showing other panels}{Shortcuts to different aspects of the program. (Battery, Maps, Trip Meter, Accelerometer, Arduino, Savings, About, Options)}



% ---------------------------------------------------------------------------
%: ----------------------- end of thesis sub-document ------------------------
% ---------------------------------------------------------------------------



%: ----------------------- contents from here ------------------------

\section{Battery}

The battery panel is the main display panel used in the user interface. It is the first panel displayed to the user when the system turns on.  It displays five important pieces of information to the user operating the vehicle. This panel listens to both the battery (TBS) and gps messages. It requires the battery messages to display the battery voltage, current and charge to the user. These are displayed using the digitelements mentioned in \ref{sec:digitelement}. It also uses a custom charge element class that exists outside of the framework to draw a battery on the center of the screen. This green bar on this battery will decrease in proportion to the charge remaining in the car, providing a quick visual indicator for how much charge is still stored. 

This panel also displays the speed in km/h, which is why it has to register to the gps module. The speed is included so the user does not have to interact with the screen while driving. This panels main purpose was to provide a quick overview to data points that are immediately of concern to the driver. 

Displayed in the top left corner is a rudimentary calculation of how much distance is remaining in the car. Range tests conducted by the REV team indicated that this distance was 80km from a full charge. To provide some security while driving, the interface assumes that the max distance the car will travel on a full charge is 70km. It uses this to calculate the distance remaining according to the formula \ref{eq:distanceremaining}

\begin{equation}
\label{eq:distanceremaining}
70=10
\end{equation}

\figuremacro{batt}{Battery state panel}{Screenshot of the battery state panel, showing the voltage, current, charge, speed (via gps) and distance remaining}



% ---------------------------------------------------------------------------
%: ----------------------- end of thesis sub-document ------------------------
% ---------------------------------------------------------------------------



%: ----------------------- contents from here ------------------------

\section{Maps}

A common, yet useful driver aid is displaying the map of the current location. This is what the maps panel does. It listens to the GPS messages to determine the position of the car. A screen-shot of this panel is shown in Figure \ref{map}. This panel uses the full area to display the current location of the vehicle on the screen. The actual position of the vehicle is centered on the middle of the screen, allowing the driver to view streets and landmarks relative to his current position. The panel features several levels of zoom, controlled by the bar on the top right corner on the screen. The plus button will zoom in, and move the slider to the right, while the minus button will zoom out, and move the slider to the left. The system supports various levels of zoom, being able to display a few buildings relative to the car or being able to display the surrounding suburbs.

\figuremacro{map}{Map display panel}{Screen-shot of the map display panel, showing the map of the current location and the map controls visible}

\subsection{Map Data}

In order to display the map images, the map data must first be obtained and stored. There are many possible methods for doing this, ranging from creating the maps as needed, or downloading them from external services and storing them in a cache. The relatively low CPU power of the eye-bot m6 makes rendering the maps on-the-fly a undesirable prospect. Open source map data for the entire planet results in a file that is 18GB in size \cite{planet_osm}. This file is much too large to store locally, and this file is actually storing a compressed version of the data. Even if it were possible for the eye-bot to store this file, via the use of external storage, it would be a large strain on the CPU to convert the street level data into viewable maps. It would also require many other pieces of software to be installed on the device, making it much more complicated to manage.

Another possibility is to use an existing Internet based map server and download the map imagery as needed. The has several downsides. Foremost it requires an Internet connection whenever to display the maps. The system does have a 3G connection installed, but this cannot be considered a dependable communication channel. It is highly likely to drop out, and is limited in coverage to the areas in which it has reception. Even without these issues, most map-servers do not allow you to download maps in bulk, as this violates their usage policies \cite{tile_usage_policy}. This makes this method undesirable as it is not suit-able for downloading maps in bulk, or as needed. Attempting to pre-download all these maps using a non-3g link would result in a violation of the usage policy.

The method chosen to obtain the map data was to pre-create create the map data using a more powerful machine. A map-server was setup and loaded with all the street data for the oceania region. For more information on the map-server setup see appendix \ref{app:mapserver}. This method overcomes the problems of the previously mentioned methods. All the processing is done on the much faster machine in advance. The area processed in advance is defined by the properties in table \ref{tab:mapbbox}. This area is depicted by the image shown in Figure \ref{mapperth}. The expanse of these maps covers all of metropolitan Perth. In future more maps can easily be processed, however this will result in more storage space being required. The current settings are a good balance between storage and expanse of data. This is because the map data will be less useful outside of the city, and the car is typically not driven any further than the pre-rendered maps. The current space usage of the map data is given in table \ref{tab:mapdata}. As this is much much larger than the internal 16MB of flash storage the eyebot has, it must be stored on external storage. Luckily, storage is now cheap, and 8GB usb thumb drive has enough performance and space to store and serve the image data.

\begin{table}[htdp]
\begin{center}
	
   \begin{tabular}{|l|l|}
        \hline
        Property           & Value           \\ \hline
        Minimum Zoom Level & 11              \\ 
        Maximum Zoom Level & 18              \\ 
        Top Left           & 115.687,-31.71  \\ 
        Bottom Right       & 116.508,-32.253 \\
        \hline
    \end{tabular}
	\caption[Properties of the pre-rendered map data]{The properties of the pre-rendered map data}
	\label{tab:mapbbox} 
\end{center}
\end{table}

\figuremacro{mapperth}{Pre-rendered map size}{This figure shows the area that has been pre-rendered for use in the map panel display}

\begin{table}[htdp]
\begin{center}
    \begin{tabular}{|l|l|}
        \hline
        Property              & Value                             \\ \hline
        Zoom Levels           & 7                                 \\ 
        Number of Files       & 435,456 \\ 
        Total Size            & 580.2 MB                          \\
        File Resolution            & 256*256 pixels                          \\  
        File Format            & Palette PNG                         \\ 
        Time taken to process & Approximately 1 Day                          \\
        \hline
    \end{tabular}
	\caption[File statistics of map data]{File statistics of map data}
	\label{tab:mapdata}
\end{center}
\end{table}

\subsection{Tiling}

As mentioned previously, the map data is much larger than the internal storage of the eye-bot. It is also much much larger than the operating systems 64MB of ram. This makes it impossible for the entire map data to be loaded inside any program to be displayed to the user. In order to overcome this limitation, the map panel only loads a small subset of the map data at a time, and displays it using a method called tiling. Rather than have one giant map that displays all of the information, and sliding this around to view the correct part, this method divides the map into many smaller maps. These smaller maps are referred to as tiles. Each tile consists of a 256 pixels wide by 256 pixels high image. These images are combined in order to display the current location, as seen in figure \ref{tiles}. By loading the adjacent tiles in all directions the screen will always have data to display.

\figuremacro{tiles}{Tiling Maps}{This figure shows the tiles that are loaded, labeled 0-8, and the area of the screen that is able to view them.}

\subsubsection{Converting GPS Co-ordinates}

The GPS outputs the current position in latitude/longitude format. This format must be converted so that the correct tiles may be loaded. To do this the GPS co-ordinates are converted to grid based co-ordinates using a mercator projection \cite{slippy_map_tilenames}. The equations to convert to this format are given by \ref{eq:gpstotiles}. It is possible to convert from a grid co-ordinate back to a GPS co-ordinate using the equations given by \ref{eq:tiletogps}. This of course will only be able to return the top left corner of the tile in GPS co-ordinates, and not the original position.

\begin{equation}
\label{eq:gpstotiles}
70=10
\end{equation}
\begin{equation}
\label{eq:tilestogps}
70=10
\end{equation}

With the GPS co-ordinate converted it is possible to locate the tile to display to the user. Rather than utilize lookup files to locate the tile, a specific folder structure is used to instantaneously load the file. This is done by storing the files in the format \emph{/zoom/tilex/tiley.png}. This allows the system to directly open the file and display it. Thus even if the entire world was loaded in the maps folder, for many different zoom levels, the display time of the current location would not change.

% ---------------------------------------------------------------------------
%: ----------------------- end of thesis sub-document ------------------------
% ---------------------------------------------------------------------------



%: ----------------------- contents from here ------------------------

\section{Trip Meter}

A useful driver aid that is common on vehicles is that of a trip meter. Traditionally this component records the distance the car has traveled since the trip meter was set. This functionality is usually a result of the simple systems in place, and is tied to the revolutions of the wheels on the vehicle. As this system has more hardware at it's disposal, the trip meter can implement more functionality than a standard trip meter, making it much more useful in examining the performance of the car. Figure \ref{trip} shows the trip meter panel being displayed on the screen.

\figuremacro{trip}{Trip Meter Panel}{This figure shows two independent trip meters and the best record speed data}

The first unique point of this trip meter, is that it contains two independent meters. This is useful as it allows the driver to evaluate the statistics of two overlapping trips. One trip meter can be used to record the distance traveled since the car was last charged, while the other can be used to record the distance traveled in the last week or month. This independence allows the operator to decide how best to use the trip meter data, resulting in a high level of flexibility. In figure \ref{trip} the trip meters are located on the left, sitting above each other.

Each trip meter records the distance traveled, the time elapsed since the meter was started, the time the car has been moving since the meter was started and calculations based on the elapsed and moving time. These statistics are displayed live to the user, but are not logged, as the logging functionality  is taken car of by a different component in the system. The trip meter panel also displays the current moving speed in the top right. Below the moving speed are the best run records in seconds. This allows the driver to have quick feedback as to how the car is performing, without having to do lots of processing on logged data.

\subsection{Distance Driven}
\label{sec:distancedriven}
The most important part of a trip meter, is the distance that the meter has recorded. This is shown on figure \ref{trip} at the bottom of each trip meter. The meter stores the distance driven internally as a double length floating point number, but displays it on the screen as a rounded integer. This is done in order improve precision for later calculations as other values will depend on the distance that has been driven. Equation \ref{eq:distancedriven} shows the the distance is calculated based on the GPS position.

\begin{align}
\label{eq:distancedriven}
\mathrm{Distance} &=\mathrm{Distance}_\mathrm{lastrun} + \mathrm{Speed}(\mathrm{Time}_\mathrm{now}-\mathrm{Time}_\mathrm{lastrun})
\end{align}

The method for working out is a continuous function that is based on the last known distance the car has traveled. This method was chosen as it does not require any information other than the last time the formula was run, and the last distance calculated. This also makes the trip meter flexible in that the distance calculation does not need to be processed at exact intervals. If messages are dropped for whatever reason, the calculation will still take place, though it will not be as accurate as it could be. The calculation will be able to cope with fluctuations in the message timing, and can adapt to the speed of the GPS being used.

The downside of this method, is that big changes in time can cause problems with the calculation. If the signal drops out for an extended period of time, such as going through a tunnel, the calculation in \ref{eq:distancedriven} would have a big margin for error. In order to prevent this, the trip meter will ignore large time differences. If two calculations are over 10 seconds apart, the result will not be trusted, and not be used in the calculation. This prevents GPS signal loss from having an adverse effect on the trip meter calculations, but does impose a limit on the trip meter.

The limitation of this method of calculating the distance driven is that it relies on the GPS messages being sent to it. If the GPS signal is lost, the distance driven will not be increased. This is a limitation imposed by the use of the GPS sensor, and cannot be avoided, as attempting to guess the distance driven while the GPS signal has been lost has a very high probability of being incorrect.

\subsection{Time Elapsed}

Another variable displayed on the trip panel is the time elapsed. This is simply the time elapsed since the trip meter was last reset. This time increments even while the GPS signal has been lost, performing a stop-watch like action on the trip. While it may seem natural to just record the time that the trip meter was started and subtract it from the current system time, this would lead to problems during the system startup. The time needs to increment even while the GPS is connecting, and must be resistant to changes in the systems internal clock. Thus the time is calculated similar to section \ref{sec:distancedriven}. The formula used to calculate the elapsed time is given in equation \ref{eq:timeelapsed}. The time elapsed is displayed as the highest element of the trip meter in Figure \ref{trip}

\begin{align}
\label{eq:timeelapsed}
\mathrm{TimeElapsed} &=\mathrm{TimeElapsed}_\mathrm{lastrun} + (\mathrm{Time}_\mathrm{now}-\mathrm{Time}_\mathrm{lastrun})
\end{align}

Much like the distance, this method of calculation depends on the last known values. This means it does not matter at the actual time the system trip started once the timer has been running. This makes it resistant to changes in the system time, and thus makes the timer more robust. 
This timer records the time elapsed on the nanosecond level, as it is used elsewhere in calculations. For display, the timer converts these values into the traditional hours, minutes, seconds format that is easy for the operator to read.

\subsection{ Moving Time}

An aspect of the cars telemetry that would be interesting to the driver is the cars moving time. This is defined as the time in which the car has spent in motion. The main use of this data point is to contrast it against the elapsed time, to highlight how long the car has spent sitting still in traffic. This variable is also useful to record for future calculations, such as working out the average speed of the trip. This element is calculated according to equation \ref{eq:timeelapsed}, except that it will not update if the current speed of the car is 0~km/h. As such this element requires the speed of the car to be processed, so it cannot be calculated when the GPS signal is lost. This fits in with the functionality defined in section \ref{sec:distancedriven}, as the trip meter will not update the moving time or distance driven if the GPS signal is lost. The moving time is displayed below the elasped time and above the distance driven in Figure \ref{trip}

\subsection{Average Speed}

When reviewing the trip meter data, it is useful to know the average speed the car was traveling during the trip. Having this information allows the driver to better understand the characteristics of the drive. This element is also easy to calculate, as the time elapsed and the distance driven are already available. Equation \ref{eq:averagespeed} shows the formula used to calculate the average speed. The calculated value is displayed to the right of the elapsed time in figure \ref{trip}.

\begin{align}
\label{eq:averagespeed}
\mathrm{Average Speed} &=\frac{\mathrm{Distance Driven}}{\mathrm{Time Elapsed}}
\end{align}

This value is calculated whenever the distance driven or elapsed time values are updated. As this value is using two calculated values, it does not need to worry about discrepancies in time or the loss of the GPS signal.

\subsection{Average Moving Speed}

The average moving speed is like the average speed. The only difference is that it uses the moving time to calculate the speed, rather than the elapsed time. This is done using the same equation \ref{eq:averagespeed}, only substituting "Time Elapsed" for "Moving Time". This value provides the average speed of the car when it was actually being driven, thus ignoring time spent waiting in traffic.

\subsection{Reset}

The final functionality of each independent trip meter is the reset button. This button resets the trip meter it is attached to. The distance driven, moving and elapsed time counters will all display zero, and the average speed calculators will display zero. As each trip meter is independent, one can be reset without affecting the other. To reset the trip meters, the driver just has to press the reset button, located below the average moving time and to the right of the distance driven in Figure \ref{trip}.

\subsection{Current Speed}

The trip meters provide statistics on where the car has been driven, but do not provide much insight into the instantaneous speed of the vehicle. As this variable is already being used in calculations it is trivial to display it to the driver. This is displayed using a simple digit element, and appears in the top right corner of Figure \ref{trip}.


% ---------------------------------------------------------------------------
%: ----------------------- end of thesis sub-document ------------------------
% ---------------------------------------------------------------------------



%: ----------------------- contents from here ------------------------

\section{Interial Measurement Unit Display Panel}

Developed in section 


\figuremacro{imu}{IMU display panel}{This figure shows the information output by the IMU daemon }



% ---------------------------------------------------------------------------
%: ----------------------- end of thesis sub-document ------------------------
% ---------------------------------------------------------------------------



%: ----------------------- contents from here ------------------------

\section{Trip Meter}



% ---------------------------------------------------------------------------
%: ----------------------- end of thesis sub-document ------------------------
% ---------------------------------------------------------------------------



%: ----------------------- contents from here ------------------------

\section{Trip Meter}



% ---------------------------------------------------------------------------
%: ----------------------- end of thesis sub-document ------------------------
% ---------------------------------------------------------------------------



%: ----------------------- contents from here ------------------------

\section{About}

% ---------------------------------------------------------------------------
%: ----------------------- end of thesis sub-document ------------------------
% ---------------------------------------------------------------------------




%: ----------------------- contents from here ------------------------

\section{Settings}

The previous features have all discussed elements of the interface that are useful to any general driver. These are features that are useful while operating the vehicle. There also exists another user that has a different set of requirements to fulfill. This user is the maintainer of the car. This user will need access to various aspects for debugging and development purposes, including the ability to easily examine and transfer the internal logs of the system. In order to help simplify the navigation of the program, these elements are located in the settings screen. A screen-shot of the settings screen is shown in figure \ref{options}.

\figuremacro{options}{Options Panel}{This figure shows the options panel, which displays various utilities that are needed by the maintainer of the vehicle.}


% ---------------------------------------------------------------------------
%: ----------------------- end of thesis sub-document ------------------------
% ---------------------------------------------------------------------------



%: ----------------------- contents from here ------------------------

\section{Debug}

The top left most button in figure \ref{options} will navigate to the debug panel. This panel is shown in figure \ref{messages}. The purpose of this panel is to display the messages that are being received from the daemon programs, see section \ref{sec:design}. This panel displays the byte level contents of each message to the user as it is received. This is used to help facilitate debugging, as the writer of the daemon is able to see exactly what is being received in the interface.

\figuremacro{messages}{Debug Messages Panel}{This figure shows the Debug panel, it allows the user to filter which messages are displayed and alternates the colour between each received message}

This display panel has various buttons that exist on the top of the screen. These buttons allow known messages to be filtered out, enabling specific debugging to be undertaken. As discussed earlier, the first 3 bytes of a message are used to filter different kinds of messages. When a button is enabled, it will allow those messages to be drawn on the screen. Multiple combinations of messages can be combined to display any combination. The buttons labeled, TBS,GPS,IMU and DIO will filter the ASCII values of the first three bytes of each message. The all button will ignore any filtering, and will allow any message to be displayed, including ones that may not have been developed or seen before.

This panel uses a ConsoleElement to display it's data as seen in section \ref{sec:consoleelement}. This allows new messages to be displayed on the bottom of the screen, and travel upwards as they become outdated. It also provides a clear distinction between messages, as seen in figure \ref{messages}. The messages alternate between red and white as they are displayed, making it easy to identify the trailing portions of each message.

\section{Network Status Display}

In order to verify that the system is working correctly, it is helpful to view the messages being sent as in the Debug Panel. This panel can provide too much information, and requires the user to understand hex in order to verify that each message is being received. In order to simplify this, a panel was developed that displays the status of each different type of message. This uses the simple boolean display element previously mentioned in section \ref{sec:diopanel}. This element will display green whenever messages are currently being received by the system. If no message is received in the previous 10 seconds, the marker will transition to red in order to illustrate this. This allows a quick view as to the health of the system, and highlights components that need to have further inspection.

\figuremacro{netstat}{Daemon Status Panel}{This figure shows the status of the various daemons, green indicates the daemon is active, red indicates a timeout}

\section{System Logs Display}

As mentioned in section \ref{sec:dmesglogging}, all aspects of this system report errors by logging them to the syslog application. This prints out the warnings and errors created to \emph{/var/log/messages}. This panel loads and displays any information logged into this file. This information may come from any of the daemons mentioned in section \ref{sec:design}, the interface itself, or even other systems that are running on the Linux installation.

\figuremacro{dmesg}{Syslog Panel}{This figure shows the System Log panel, this displays the most recent entries in \emph{/var/log/messages}}

An important aspect of this panel is that it only displays the most recent information that has been logged. This makes it compatible with the ConsoleElement described in section \ref{sec:consoleelement}. The Console Element will automatically take care of displaying the last pieces of information it has received. This would produce the desired functionality, however further optimization needed to be undertaken.

The file \emph{/var/log/messages}, is logged to by all applications since the system started. Thus it grows in size the longer the system has been operational. Without optimizing the way this file is read, it would take O(n) time to display the newest entries in the file on the Console Element, where n is the number of entries in the file. This is not acceptable as the time taken to display the logs will increase the longer the application is running. In order to fix this problem, the file is read from the end of the file minus a 2048 byte offset.  The maximum width that can be displayed is 60 characters, and the maximum possible lines that can be shown on the screen is 20. This yields the result of 1200 possible characters as seen in equation \ref{eq:maximumdisplaycharacters}. This size is roughly doubled to 2048, in order to ensure more than enough lines are read for each update. This method of reading the last 2048 characters and finding the unique lines inside them is not affected by the file size, so the time taken to update the display is now O(1). The update is not dependent on the number of lines in \emph{/var/log/messages}

\begin{align}
\label{eq:maximumdisplaycharacters}
\mathrm{Maximum\ Characters} &= \mathrm{Characters\ per\ line}* \mathrm{Number\ of \ Lines} \\
&= 20*60 \notag \\
&=1200 \notag 
\end{align}

As the actual log file may be updated very quickly the screen must not update automatically. This is to stop the screen transitioning while the user may be reading something on it. As such the panel does not support any automatically updating functionality. In order to update the screen, a system like that in section \ref{sec:screenabout} is used. A invisible button is created to span the whole display. When pressed, this button will read the \emph{/var/log/messages} file and update the screen accordingly. This allows for the maximum utilization of space possible, as no space needs to be created to display a refresh button. The messages will also be updated when this panel becomes the active panel displayed on the screen.


% ---------------------------------------------------------------------------
%: ----------------------- end of thesis sub-document ------------------------
% ---------------------------------------------------------------------------



%: ----------------------- contents from here ------------------------

\section{Copy Log Files}

Discussed in section \ref{sec:filelogger} is a utility that records the telemetry data locally. In order to make the files created by this logger easier to access, the user interface provides a mechanism for them to be copied onto an external usb drive. Figure \ref{copy} shows the display asking for user confirmation to copy the log files. If the user confirms the action, the program will navigate to the directory in which the log files are stored, and copy them byte by byte to the external drive. Once completed the screen will display the figure \ref{copycompleted}. 
\figuremacro{copy}{Copy Popup}{Popup asking the user if they wish to copy log files to the an external drive}

If an error occurs, the system will log the results to the in-built system logger, syslog. This will allow the results to be viewable at \emph{/var/log/messages}, and thus can be viewed by the Syslog panel. While copying, the interface will display the progress of the action. It does this by indicating the amount of files to be copied, and the current file it is copying across. This is implemented so the user can be sure the system has not hung or encountered another error.

\figuremacro{copycompleted}{Copy Completed Popup}{This popup confirms that all the log files have been correctly copied to the usb drive}

\section{Delete Log Files}

In order to prevent the storage media from filling up, and to help organize the files when they are copied over the interface provides a mechanism to delete all the log files. Figure \ref{delete} shows a popup asking the user if they wish to delete the files. This functionality functions much the same way as copy, except that it does not require an external drive. It will simply open the folder and delete the files one by one. This action will also display its progress, and will provide the user with a message once the action is complete.

\figuremacro{delete}{Delete Log Files Popup}{This asks the user if they want to delete all the log files stored on the system}

% ---------------------------------------------------------------------------
%: ----------------------- end of thesis sub-document ------------------------
% ---------------------------------------------------------------------------


\section{Exit Program}

In order to allow other programs that use the various hardware, such as the screen or the touch driver, to run the current application must be exited. The system facilitates this be placing an exit action on the options panel. When this button is pressed, the popup shown in figure \ref{exit} will be displayed. This popup confirms that the user wishes to exit the application. If the user agrees, the program will exit and the screen will be set to black. If the user cancels, the system will return to the previously displayed panel, allowing normal operation to resume.

\figuremacro{exit}{Exit Popup}{This popup confirms that the user wants to exit the running application}




% ---------------------------------------------------------------------------
%: ----------------------- end of thesis sub-document ------------------------
% ---------------------------------------------------------------------------

