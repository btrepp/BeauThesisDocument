% this file is called up by thesis.tex
% content in this file will be fed into the main document

%: ----------------------- name of chapter  -------------------------
\chapter{Interface} % top level followed by section, subsection


%: ----------------------- paths to graphics ------------------------

% change according to folder and file names
\ifpdf
    \graphicspath{{6/figures/PNG/}{6/figures/PDF/}{6/figures/}}
\else
    \graphicspath{{6/figures/EPS/}{6/figures/}}
\fi

%: ----------------------- contents from here ------------------------


\section{Layout}

An important aspect of user interfaces is that they must be visually impressive [GO CITE?]. This keeps the user interested in the product, and helps new users want to learn how to operate the device. The work done in the windowing toolkit allowed a much more visually impressive layout to be developed. This was achieved through the use of transparencies in order to overlay different elements on top of each other.

\subsection{Background}

As the device does not have enough power to dynamically render any sort of complex images or motions, a pre-rendered static background was used. This background was designed on another pc, and converted for use on the embedded system. It consists of two main regions. The first area is the main display region. This section is the largest of the space and exists so the current data being displayed can be laid over it. The second region is the navigation bar region. This area is at the bottom of the image, it uses a more uniform texture in order to contrast the main data display area above it. The background itself is rendered in black and white. The black and white styling allows important information to contrast against it easily. This helps highlight the information and user interface actions present to the user, and avoids a simple single-colour background on the user interface.


%: ----------------------- contents from here ------------------------

\section{Overview Panel}

To facilitate easy navigation a panel was designed that shows all the different panels available. This panel is shown in figure \ref{app}. It contains 8 different aspects of the program, with room available for 12. This will allow the system to be extended in the future. 

\figuremacro{app}{Panel showing other panels}{Shortcuts to different aspects of the program. (Battery, Maps, Trip Meter, Accelerometer, Arduino, Savings, About, Options)}



% ---------------------------------------------------------------------------
%: ----------------------- end of thesis sub-document ------------------------
% ---------------------------------------------------------------------------



%: ----------------------- contents from here ------------------------

\section{Battery}


\figuremacro{batt}{Panel showing other panels}{Shortcuts to different aspects of the program. (Battery, Maps, Trip Meter, Accelerometer, Arduino, Savings, About, Options)}



% ---------------------------------------------------------------------------
%: ----------------------- end of thesis sub-document ------------------------
% ---------------------------------------------------------------------------



%: ----------------------- contents from here ------------------------

\section{Maps}

A common, yet useful driver aid is displaying the map of the current location. This is what the maps panel does. It listens to the GPS messages to determine the position of the car. A screen-shot of this panel is shown in Figure \ref{map}. This panel uses the full area to display the current location of the vehicle on the screen. The actual position of the vehicle is centered on the middle of the screen, allowing the driver to view streets and landmarks relative to his current position. The panel features several levels of zoom, controlled by the bar on the top right corner on the screen. The plus button will zoom in, and move the slider to the right, while the minus button will zoom out, and move the slider to the left. The system supports various levels of zoom, being able to display a few buildings relative to the car or being able to display the surrounding suburbs.

\figuremacro{map}{Map display panel}{Screen-shot of the map display panel, showing the map of the current location and the map controls visible}

\subsection{Map Data}

In order to display the map images, the map data must first be obtained and stored. There are many possible methods for doing this, ranging from creating the maps as needed, or downloading them from external services and storing them in a cache. The relatively low CPU power of the eye-bot m6 makes rendering the maps on-the-fly a undesirable prospect. Open source map data for the entire planet results in a file that is 18GB in size \cite{planet_osm}. This file is much too large to store locally, and this file is actually storing a compressed version of the data. Even if it were possible for the eye-bot to store this file, via the use of external storage, it would be a large strain on the CPU to convert the street level data into viewable maps. It would also require many other pieces of software to be installed on the device, making it much more complicated to manage.

Another possibility is to use an existing Internet based map server and download the map imagery as needed. The has several downsides. Foremost it requires an Internet connection whenever to display the maps. The system does have a 3G connection installed, but this cannot be considered a dependable communication channel. It is highly likely to drop out, and is limited in coverage to the areas in which it has reception. Even without these issues, most map-servers do not allow you to download maps in bulk, as this violates their usage policies \cite{tile_usage_policy}. This makes this method undesirable as it is not suit-able for downloading maps in bulk, or as needed. Attempting to pre-download all these maps using a non-3g link would result in a violation of the usage policy.

The method chosen to obtain the map data was to pre-create create the map data using a more powerful machine. A map-server was setup and loaded with all the street data for the oceania region. For more information on the map-server setup see appendix \ref{app:mapserver}. This method overcomes the problems of the previously mentioned methods. All the processing is done on the much faster machine in advance. The area processed in advance is defined by the properties in table \ref{tab:mapbbox}. This area is depicted by the image shown in Figure \ref{mapperth}. The expanse of these maps covers all of metropolitan Perth. In future more maps can easily be processed, however this will result in more storage space being required. The current settings are a good balance between storage and expanse of data. This is because the map data will be less useful outside of the city, and the car is typically not driven any further than the pre-rendered maps. 

\begin{table}
\begin{center}
	\label{tab:mapbbox} 
   \begin{tabular}{|l|l|}
        \hline
        Property           & Value           \\ \hline
        Minimum Zoom Level & 11              \\ 
        Maximum Zoom Level & 18              \\ 
        Top Left           & 115.687,-31.71  \\ 
        Bottom Right       & 116.508,-32.253 \\
        \hline
    \end{tabular}
\end{center}
\end{table}

\figuremacro{mapperth}{Pre-rendered map size}{This figure shows the area that has been pre-rendered for use in the map panel display}


% ---------------------------------------------------------------------------
%: ----------------------- end of thesis sub-document ------------------------
% ---------------------------------------------------------------------------



%: ----------------------- contents from here ------------------------

\section{Trip Meter}

A useful driver aid that is common on vehicles is that of a trip meter. Traditionally this component records the distance the car has traveled since the trip meter was set. This functionality is usually a result of the simple systems in place, and is tied to the revolutions of the wheels on the vehicle. As this system has more hardware at it's disposal, the trip meter can implement more functionality than a standard trip meter, making it much more useful in examining the performance of the car. Figure \ref{trip} shows the trip meter panel being displayed on the screen. An important note of this panel is that all calculations are done whether the panel is being displayed or not.

\figuremacro{trip}{Trip Meter Panel}{This figure shows two independent trip meters and the best record speed data}

The first unique point of this trip meter, is that it contains two independent meters. This is useful as it allows the driver to evaluate the statistics of two overlapping trips. One trip meter can be used to record the distance traveled since the car was last charged, while the other can be used to record the distance traveled in the last week or month. This independence allows the operator to decide how best to use the trip meter data, resulting in a high level of flexibility. In figure \ref{trip} the trip meters are located on the left, sitting above each other.

Each trip meter records the distance traveled, the time elapsed since the meter was started, the time the car has been moving since the meter was started and calculations based on the elapsed and moving time. These statistics are displayed live to the user, but are not logged, as the logging functionality  is taken car of by a different component in the system. The trip meter panel also displays the current moving speed in the top right. Below the moving speed are the best run records in seconds. This allows the driver to have quick feedback as to how the car is performing, without having to do lots of processing on logged data.

\subsection{Distance Driven}
\label{sec:distancedriven}
The most important part of a trip meter, is the distance that the meter has recorded. This is shown on figure \ref{trip} at the bottom of each trip meter. The meter stores the distance driven internally as a double length floating point number, but displays it on the screen as a rounded integer. This is done in order improve precision for later calculations as other values will depend on the distance that has been driven. Equation \ref{eq:distancedriven} shows the the distance is calculated based on the GPS position.

\begin{align}
\label{eq:distancedriven}
\mathrm{Distance} &=\mathrm{Distance}_\mathrm{lastrun} + \mathrm{Speed}(\mathrm{Time}_\mathrm{now}-\mathrm{Time}_\mathrm{lastrun})
\end{align}

The method for working out is a continuous function that is based on the last known distance the car has traveled. This method was chosen as it does not require any information other than the last time the formula was run, and the last distance calculated. This also makes the trip meter flexible in that the distance calculation does not need to be processed at exact intervals. If messages are dropped for whatever reason, the calculation will still take place, though it will not be as accurate as it could be. The calculation will be able to cope with fluctuations in the message timing, and can adapt to the speed of the GPS being used.

The downside of this method, is that big changes in time can cause problems with the calculation. If the signal drops out for an extended period of time, such as going through a tunnel, the calculation in \ref{eq:distancedriven} would have a big margin for error. In order to prevent this, the trip meter will ignore large time differences. If two calculations are over 10 seconds apart, the result will not be trusted, and not be used in the calculation. This prevents GPS signal loss from having an adverse effect on the trip meter calculations, but does impose a limit on the trip meter.

The limitation of this method of calculating the distance driven is that it relies on the GPS messages being sent to it. If the GPS signal is lost, the distance driven will not be increased. This is a limitation imposed by the use of the GPS sensor, and cannot be avoided, as attempting to guess the distance driven while the GPS signal has been lost has a very high probability of being incorrect.

\subsection{Time Elapsed}

Another variable displayed on the trip panel is the time elapsed. This is simply the time elapsed since the trip meter was last reset. This time increments even while the GPS signal has been lost, performing a stop-watch like action on the trip. While it may seem natural to just record the time that the trip meter was started and subtract it from the current system time, this would lead to problems during the system startup. The time needs to increment even while the GPS is connecting, and must be resistant to changes in the systems internal clock. Thus the time is calculated similar to section \ref{sec:distancedriven}. The formula used to calculate the elapsed time is given in equation \ref{eq:timeelapsed}. The time elapsed is displayed as the highest element of the trip meter in Figure \ref{trip}

\begin{align}
\label{eq:timeelapsed}
\mathrm{TimeElapsed} &=\mathrm{TimeElapsed}_\mathrm{lastrun} + (\mathrm{Time}_\mathrm{now}-\mathrm{Time}_\mathrm{lastrun})
\end{align}

Much like the distance, this method of calculation depends on the last known values. This means it does not matter at the actual time the system trip started once the timer has been running. This makes it resistant to changes in the system time, and thus makes the timer more robust. 
This timer records the time elapsed on the nanosecond level, as it is used elsewhere in calculations. For display, the timer converts these values into the traditional hours, minutes, seconds format that is easy for the operator to read.

\subsection{ Moving Time}

An aspect of the cars telemetry that would be interesting to the driver is the cars moving time. This is defined as the time in which the car has spent in motion. The main use of this data point is to contrast it against the elapsed time, to highlight how long the car has spent sitting still in traffic. This variable is also useful to record for future calculations, such as working out the average speed of the trip. This element is calculated according to equation \ref{eq:timeelapsed}, except that it will not update if the current speed of the car is 0~km/h. As such this element requires the speed of the car to be processed, so it cannot be calculated when the GPS signal is lost. This fits in with the functionality defined in section \ref{sec:distancedriven}, as the trip meter will not update the moving time or distance driven if the GPS signal is lost. The moving time is displayed below the elasped time and above the distance driven in Figure \ref{trip}

\subsection{Average Speed}

When reviewing the trip meter data, it is useful to know the average speed the car was traveling during the trip. Having this information allows the driver to better understand the characteristics of the drive. This element is also easy to calculate, as the time elapsed and the distance driven are already available. Equation \ref{eq:averagespeed} shows the formula used to calculate the average speed. The calculated value is displayed to the right of the elapsed time in figure \ref{trip}.

\begin{align}
\label{eq:averagespeed}
\mathrm{Average Speed} &=\frac{\mathrm{Distance Driven}}{\mathrm{Time Elapsed}}
\end{align}

This value is calculated whenever the distance driven or elapsed time values are updated. As this value is using two calculated values, it does not need to worry about discrepancies in time or the loss of the GPS signal.

\subsection{Average Moving Speed}

The average moving speed is like the average speed. The only difference is that it uses the moving time to calculate the speed, rather than the elapsed time. This is done using the same equation \ref{eq:averagespeed}, only substituting "Time Elapsed" for "Moving Time". This value provides the average speed of the car when it was actually being driven, thus ignoring time spent waiting in traffic.

\subsection{Reset}

The final functionality of each independent trip meter is the reset button. This button resets the trip meter it is attached to. The distance driven, moving and elapsed time counters will all display zero, and the average speed calculators will display zero. As each trip meter is independent, one can be reset without affecting the other. To reset the trip meters, the driver just has to press the reset button, located below the average moving time and to the right of the distance driven in Figure \ref{trip}.

\subsection{Current Speed}

The trip meters provide statistics on where the car has been driven, but do not provide much insight into the instantaneous speed of the vehicle. As this variable is already being used in calculations it is trivial to display it to the driver. This is displayed using a simple digit element, and appears in the top right corner of Figure \ref{trip}.

\subsection{Time Trial Data}

A common metric in measuring the performance of cars is to measure how long the car takes to achieve a certain speed. Usually this requires expensive equipment in order to accurately measure the time and speed data. As the information exists inside the trip meter in some form already, it is useful to display a less accurate version of this time trial data. By recording the time it takes to reach 50 or 100 km/h the operator is able to have quick feedback on the performance, without having to setup lots of equipment. The flow chart of this calculation is given by Figure \ref{timetrialflowchart}. The results of this are displayed below the current speed in Figure \ref{trip}

\figuremacro{timetrialflowchart}{Time Trial Data flow chart}{The program flow used to calculate the 0-50 km/h and 0-100 km/h time trial data}

An important condition on this flow chart is that the zero time must be set before any calculations are performed. The zero time is the time at which the car was last traveling 0~km/h. This time will be reset whenever the car is stopped, so the calculation requires no input from the user. Also present in the flow diagram is that the system will display the best time recorded. If there is a previous best time, and a new one is achieved, the new one will automatically be displayed. This allows the user to ignore the trip meter panel entirely, and be able to trigger and record some performance data on normal drives.

\subsection{Persistance}

TODO: talk about file format and persistance


% ---------------------------------------------------------------------------
%: ----------------------- end of thesis sub-document ------------------------
% ---------------------------------------------------------------------------



%: ----------------------- contents from here ------------------------

\section{Arduino Digital Input Module}
\label{sec:arduino}

The car has access to many physical inputs that would be useful to monitor and record. Variables such as the state of the air-conditioning or the radio are useful aspects to monitor. These are currently exposed via bare wires inside the vehicle. As these signals are simple digital logic, they need a way to interface with the controller in order to be used in the system.

Previously the in-built FPGA on the Eye-bot was used to accomplish this end \cite{thesis_varma}. This method requires complex cabling to the inside of the Eye-bot. Use of the FPGA also requires special code to be written to interface with the FPGA, which will be different for different kinds of FPGA devices. This makes maintaining this design difficult, and ties the code in with the specific Eye-bot it was developed for.

Investigation was done into a top16 digital input and output module. This module has 8 digital input lines, and 8 digital output lines. This allows for a total of 8 inputs to be read. While this would be enough to satisfy the input variables, it suffers from some drawbacks. This board was found to use an FTDI chip for usb serial communication. This chip appears as a serial port when the correct drivers are present. Sadly the FTDI drivers for the older Linux kernel were not stable, and it was not possible to use this board in the system.

As the current methods were unsuitable to fulfill this role, a new digital interface system was developed in order to satisfy the requirements needed. This new board would have digital inputs, analogue voltage inputs, and be able to read other signals generated by the car.

\subsection{Speedometer and Tachometer}

A desired feature would be to record more complicated signals from the car, such as the current speed. While the GPS can provide speed readings, a more accurate source of speed data is available. This source is the cars built-in speedometer. The speedometer and tachometer use hall effect sensors in order to read the rotational rate of the gearbox and motor respectively. 

\figuremacro{speedvsfrequency}{Speed (km/h) vs Frequency (Hz)}{}
\figuremacro{rpmvsfrequency}{RPM vs Frequency (Hz)}{}

Figures \ref{speedvsfrequency} and \ref{rpmvsfrequency} show the output on the in-built dash in response to various pulse train frequencies. Both these graphs show that the relationship between the frequency and desired variable are highly linear. Thus it is possible to count the amount of pulses that have transpired and perform a calculation in order to determine the cars current speed.

\subsection{Hardware}

The hardware used to accomplish this task is an Arduino Uno compatible board. This board provides 14 Digital IO pins and 6 Analogue Input pins \cite{arduinospecs}. The Atmega chip inside the board also has inbuilt timers, which can be used to implement the frequency component of the requirements. The board is quite small in size, and only needs 5v to run. Figure \ref{ArduinoUno} shows the Arduino Uno board.

\figuremacroW{ArduinoUno}{Arduino Uno}{}{0.5}

\subsection{ Drivers}

One advantage of the Arduino Uno board over other Arduino boards is that it implements the cdc-acm device drivers that were used in section \ref{sec:gpsdrivers}. The fixes to the drivers implemented previously allow the Arduino Uno to work with the system without any hassle. 

\subsection {Design}

\subsubsection{Arduino}

The program designed for the Arduino uses the internal interrupts of the Arduino in order to keep track of time and the amount of pulse that have occurred. The Arduino contains an ATmega328 \cite{arduinospecs}. This chip contains internal circuitry that is able to count the number of pulses independently of the current system clock. There are two pins available on the micro-controller to facilitate this functionality, and two frequencies the system is interested in recording.

\begin{table}
\begin{center}
    \begin{tabular}{|l|l|}
        \hline
        Property & Value \\ \hline
        Baud Rate    & 115200 \\
        Data Bits    & 8    \\ 
        Stop Bits    & 1    \\ 
        Parity       & None \\ 
        Flow Control & None \\
        \hline
    \end{tabular}
	\caption{Connection settings for Arduino}
\end{center}
\end{table}

The main logic of the Arduino program is fired in an interrupt that fires at a frequency of 125Hz. This interrupt checks the overflow of the 8 bit counter, and stores the result so it can be used later on in the calculation. This allows the program to work with larger numbers than a 8 bit byte can contain. Every 125 cycles of this interrupt, or every one second, the system calculates the number of pulses that have occurred since the last second, and stores it in a different memory location. The program also sets a flag inside the micro-controller, signaling that the data is ready to be sent via a serial link.

When the serial link is signaled, it will read all the digital inputs and analogue inputs as well as the calculated pulses, and transmit them to the device on the other end. As this is a serial stream, it must have some values that are reserved in order to synchronize the data stream. The same method as used in section \ref{sec:expertprotocol}. This uses the same header byte (0xFF) and allows for 7 bits of data to be transmitted per byte. Rather than sending multiple messages, the Arduino sends everything in one message, that is transmitted at 1Hz.

\begin{table}
    \begin{tabular}{*{7}{|l}|}
        \hline
        Byte 1 & Byte 2 & Byte 3 & Byte 4 & Byte 5 & Byte 6  & Byte 7 \\ \hline \hline
        Header & Count & 7 Bits DIO & 5 bits DIO & Analog0 MSB  & Analog0 LSB & Analog 1 MSB  \\ 
        \hline
    \end{tabular}
 \begin{tabular}{*{8}{|l}|}
        \hline
        Byte 8 & Byte 9 & Byte 10 & Byte 11 & Byte 12 & Byte 13 & Byte 14 & Byte 15\\ \hline \hline
       Analog 1 LSB & \multicolumn{4}{|c|}{Frequency 0} & \multicolumn{3}{|c|}{Frequency 1} \\
        \hline
    \end{tabular}
\begin{tabular}{*{8}{|l}|}
        \hline
         Byte 16 & Byte 17 & Byte 18 & Byte 19 & Byte 20 & Byte 21 & Byte 22 \\ \hline \hline
        \multicolumn{1}{|c|}{Frequency 1} & \multicolumn{6}{|c|}{Reserved} & Trailer \\
        \hline
    \end{tabular}
	\caption{Network protocol for the distributed system}
	\label{tab:networkprotocol}
\end{table}

\subsubsection{Host side}

On the host side, the program functions similarly to that of the GPS program, such as in Figure \ref{gpsflowchart}. This module of course, will not set the system date, but otherwise they perform similarly, attempting to read as much data as possible into a buffer, and then finding the correct section to synchronize on.







% ---------------------------------------------------------------------------
%: ----------------------- end of thesis sub-document ------------------------
% ---------------------------------------------------------------------------



%: ----------------------- contents from here ------------------------

\section{Cost Panel}

\figuremacro{money}{Savings Panel}{}


% ---------------------------------------------------------------------------
%: ----------------------- end of thesis sub-document ------------------------
% ---------------------------------------------------------------------------



%: ----------------------- contents from here ------------------------

\section{About}

% ---------------------------------------------------------------------------
%: ----------------------- end of thesis sub-document ------------------------
% ---------------------------------------------------------------------------







% ---------------------------------------------------------------------------
%: ----------------------- end of thesis sub-document ------------------------
% ---------------------------------------------------------------------------

