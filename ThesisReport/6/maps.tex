
%: ----------------------- contents from here ------------------------

\section{Maps}
\label{sec:maps}
A common, yet useful driver aid is displaying the map of the current location. This is what the maps panel does. It listens to the GPS messages to determine the position of the car. A screen-shot of this panel is shown in Figure \ref{map}. This panel uses the full area to display the current location of the vehicle on the screen. The actual position of the vehicle is centered on the middle of the screen, allowing the driver to view streets and landmarks relative to his current position. The panel features several levels of zoom, controlled by the bar on the top right corner on the screen. The plus button will zoom in, and move the slider to the right, while the minus button will zoom out, and move the slider to the left. The system supports various levels of zoom, being able to display a few buildings relative to the car or being able to display the surrounding suburbs.

\figuremacro{map}{Map display panel}{Screen-shot of the map display panel, showing the map of the current location and the map controls visible}

The controls visible on Figure \ref{map} are not always required by the operator. They can obscure parts of the map in which the driver may be interested in. This can be a problem as the driver should not be expected to adjust the position of the car, in order to see a part of the map that is hidden by the UI controls. In order to prevent this problem, the controls can be hidden by tapping any area of the screen that is not occupied by a button. Doing this will hide all the navigation elements and just display the map as shown in \ref{maphidden}.

\figuremacro{maphidden}{Map display panel with hidden controls}{Screen-shot of the map display panel, showing the map of the current location and the map controls hidden}.


\subsection{Map Data}

In order to display the map images, the map data must first be obtained and stored. There are many possible methods for doing this, ranging from creating the maps as needed, or downloading them from external services and storing them in a cache. The relatively low CPU power of the eye-bot m6 makes rendering the maps on-the-fly a undesirable prospect. Open source map data for the entire planet results in a file that is 18GB in size \cite{planet_osm}. This file is much too large to store locally, and this file is actually storing a compressed version of the data. Even if it were possible for the eye-bot to store this file, via the use of external storage, it would be a large strain on the CPU to convert the street level data into viewable maps. It would also require many other pieces of software to be installed on the device, making it much more complicated to manage.

Another possibility is to use an existing Internet based map server and download the map imagery as needed. The has several downsides. Foremost it requires an Internet connection whenever to display the maps. The system does have a 3G connection installed, but this cannot be considered a dependable communication channel. It is highly likely to drop out, and is limited in coverage to the areas in which it has reception. Even without these issues, most map-servers do not allow you to download maps in bulk, as this violates their usage policies \cite{tile_usage_policy}. This makes this method undesirable as it is not suit-able for downloading maps in bulk, or as needed. Attempting to pre-download all these maps using a non-3g link would result in a violation of the usage policy.

The method chosen to obtain the map data was to pre-create create the map data using a more powerful machine. A map-server was setup and loaded with all the street data for the Oceania region. For more information on the map-server setup see appendix \ref{app:sec:mapserver}. This method overcomes the problems of the previously mentioned methods. All the processing is done on the much faster machine in advance. The area processed in advance is defined by the properties in table \ref{tab:mapbbox}. This area is depicted by the image shown in Figure \ref{mapperth}. The expanse of these maps covers all of metropolitan Perth. In future more maps can easily be processed, however this will result in more storage space being required. The current settings are a good balance between storage and expanse of data. This is because the map data will be less useful outside of the city, and the car is typically not driven any further than the pre-rendered maps. The current space usage of the map data is given in table \ref{tab:mapdata}. As this is much much larger than the internal 16MB of flash storage the eyebot has, it must be stored on external storage. Luckily, storage is now cheap, and 8GB usb thumb drive has enough performance and space to store and serve the image data.

\begin{table}[htdp]
\begin{center}
	
   \begin{tabular}{|l|l|}
        \hline
        Property           & Value           \\ \hline
        Minimum Zoom Level & 11              \\ 
        Maximum Zoom Level & 18              \\ 
        Top Left           & 115.687,-31.71  \\ 
        Bottom Right       & 116.508,-32.253 \\
        \hline
    \end{tabular}
	\caption[Properties of the pre-rendered map data]{The properties of the pre-rendered map data}
	\label{tab:mapbbox} 
\end{center}
\end{table}

\figuremacro{mapperth}{Pre-rendered map size}{This figure shows the area that has been pre-rendered for use in the map panel display}

\begin{table}[htdp]
\begin{center}
    \begin{tabular}{|l|l|}
        \hline
        Property              & Value                             \\ \hline
        Zoom Levels           & 7                                 \\ 
        Number of Files       & 435,456 \\ 
        Total Size            & 580.2 MB                          \\
        File Resolution            & 256*256 pixels                          \\  
        File Format            & Palette PNG                         \\ 
        Time taken to process & Approximately 1 Day                          \\
        \hline
    \end{tabular}
	\caption[File statistics of map data]{File statistics of map data}
	\label{tab:mapdata}
\end{center}
\end{table}

\subsection{Tiling}

As mentioned previously, the map data is much larger than the internal storage of the eye-bot. It is also much much larger than the operating systems 64MB of ram. This makes it impossible for the entire map data to be loaded inside any program to be displayed to the user. In order to overcome this limitation, the map panel only loads a small subset of the map data at a time, and displays it using a method called tiling. Rather than have one giant map that displays all of the information, and sliding this around to view the correct part, this method divides the map into many smaller maps. These smaller maps are referred to as tiles. Each tile consists of a 256 pixels wide by 256 pixels high image. These images are combined in order to display the current location, as seen in figure \ref{tiles}. By loading the adjacent tiles in all directions the screen will always have data to display.

\figuremacro{tiles}{Tiling Maps}{This figure shows the tiles that are loaded, labeled 0-8, and the area of the screen that is able to view them.}

\subsubsection{Converting GPS Co-ordinates}

The GPS outputs the current position in latitude/longitude format. This format must be converted so that the correct tiles may be loaded. To do this the GPS co-ordinates are converted to grid based co-ordinates using a Mercator projection \cite{slippy_map_tilenames}. The equations to convert to this format are given by \ref{eq:gpstotiles} \cite{slippy_map_tilenames}. It is possible to convert from a grid co-ordinate back to a GPS co-ordinate using the equations given by \ref{eq:tilestogps} \cite{slippy_map_tilenames}. This of course will only be able to return the top left corner of the tile in GPS co-ordinates, and not the original position.

\begin{align}
\label{eq:gpstotiles}
\mathrm{Tile\ X} = \lfloor(\frac{\mathrm{longitude}+180.0}{360}* 2^{\mathrm{zoom}})\rfloor 
\end{align}
\begin{align}
\mathrm{Tile\ Y} = \lfloor{{\left(1- \frac{\ln \left(\tan(\frac{\pi*\mathrm{latitude}}{180})+\frac{1}{\cos(\frac{\mathrm{latitude}*\pi}{180})}\right)}{\pi} \right  )}*2^{\mathrm{zoom}-1}}\rfloor
\end{align}

\begin{align}
\label{eq:tilestogps}
\mathrm{Longitude} = \frac{\mathrm{X}*360}{2^{\mathrm{zoom}}} -180 \\
\end{align}
\begin{align}
n = \frac{ \pi- 2.0*\pi*\mathrm{Y}}{2^{\mathrm{zoom}}} \\
\mathrm{Latitude} = \frac{180*\tan^-1(0.5*(e^n-e^-n))}{\pi} 
\end{align}

With the GPS co-ordinate converted it is possible to locate the tile to display to the user. Rather than utilize lookup files to locate the tile, a specific folder structure is used to instantaneously load the file. This is done by storing the files in the format \emph{/zoom/tilex/tiley.png}. This allows the system to directly open the file and display it. Thus even if the entire world was loaded in the maps folder, for many different zoom levels, the display time of the current location would not change.

\subsection{Palleted File Format}

The amount of storage required for all these tiles in an uncompressed format is shown in equation \ref{eq:mapstorage} using the data obtained in table \ref{tab:mapdata}. This amounts to approximately 81 gigabytes of information at all the various zoom levels. This size is much too large to store and a small usb drive. It would be possible to store this data on a mechanical disk, but this is much more costly than a cheaper flash based drive, and would require much more power to run.
In order to solve this problem, a compressed file format is used. The PNG file format results in a much more smaller filesize per tile. The average filesize of the final compresed files is approximately 1.3~kB. This is a large contrast to the size of one file in an uncompressed format, being 192~kB. PNG uses various compression techniques to greatly reduce the filesize of the tiles.

\begin{align}
\label{eq:mapstorage}
Total Space Required &=\frac {\mathrm{height} * \mathrm{width} *\mathrm{image}* {\mathrm{number\ of\ tiles}}} {{ \mathrm{bytes\ in\ a\ megabyte}}} \\
&= \frac{256*256*3*435456}{1048676} \notag \\
&= 81648~MB \notag
\end{align}

The main technique that is utilized in this system is the use of pallettes. Palleting restricts the colour space in the image, in order to save space. The map images do not utilized the full 24bit colour space that the screen is capable of displaying as the maps need to be visually clean, they only use a handful of colours to represent the area. Palleting works by defining the sets of colours that will be used in the image beforehand, or the pallette of this image. Consider an image with only two colours. If each colour in the pallette is given a value, each pixel in the image can be represented by that value. Hence each pixel can be represented by 1 bit, rather than 3 bytes (24 bits). This is a simple example, but it highlights how pallettes can be used to save space.
The downside to using this technique, is that it is slightly slower than using raw image data. As the pallette is represented in a different format, it must be loaded and converted to the unprocessed data when needed. This takes some time compared to just reading the raw data. This trade-off is worthwhile however, as the space saving more than makes up for the extra CPU time to load the images. The method also has the advantage that the tile images can be easily read or modified by consumer software on other machines, PNG files are very common and can be natively viewed on all full-blown operating systems.


\subsection{Sliding Maps}

The last feature that the map panel implements is centering the vehicles position on the screen. This means that the screen will slide around in response to changes in the vehicles position. It does this by using a modified version of the equations in \ref{eq:gpstotiles}. These modified equation works takes the same inputs, but returns the position of the car inside the grid co-ordinate as a floating point number. This return result has the range $0 \le result<1$. A return value of 0 indicates that the top right corner of the image should be placed in the center of the screen. A return value of 1 is never possible, as this would be 0 on a different tile. This allows the tiles to be moved around as the car moves, providing sleek graphical feedback to the operator of the vehicle.


% ---------------------------------------------------------------------------
%: ----------------------- end of thesis sub-document ------------------------
% ---------------------------------------------------------------------------

