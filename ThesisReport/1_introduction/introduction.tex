
% this file is called up by thesis.tex
% content in this file will be fed into the main document

%: ----------------------- introduction file header -----------------------
\chapter{Introduction}

% the code below specifies where the figures are stored
\ifpdf
    \graphicspath{{1_introduction/figures/PNG/}{1_introduction/figures/PDF/}{1_introduction/figures/}}
\else
    \graphicspath{{1_introduction/figures/EPS/}{1_introduction/figures/}}
\fi

% ----------------------------------------------------------------------
%: ----------------------- introduction content ----------------------- 
% ----------------------------------------------------------------------



%: ----------------------- HELP: latex document organisation
% the commands below help you to subdivide and organise your thesis
%    \chapter{}       = level 1, top level
%    \section{}       = level 2
%    \subsection{}    = level 3
%    \subsubsection{} = level 4
% note that everything after the percentage sign is hidden from output



\section{The REV Group}

The Renewable Energy Vehicle project aims to explore the use of electric engines in vehicles in place of the traditional internal combustion engine. This can provide many benefits; including helping remove societies dependency on oil and reducing emissions from personal transportation. The Hyundai Getz is designed to be a substitute for economical cars. This car is designed to fit into the “daily driver” market. This market needs smaller cars, with short range but cheap running costs, as well as being enjoyable to drive. 

\subsection{Motivations} 
 
Many factors motivate research into electric based vehicles. The two most prominent are the emissions of traditional vehicles and the continuing rise in cost of fuels.

\subsubsection{Pollution} % subsection headings are again smaller than section names
% lead

Electric vehicles are advantageous over traditional ICE vehicles as they operate with zero emissions while operating . These vehicles have no exhaust, so therefore have no emissions. While this does not make them completely pollutant free, it does help limit and control the emissions being produced by the act of transport. It is important to remember when discussing electric vehicles that the components must be manufactured using industrial processes and the act of generation electricity. This does not making them truly carbon neutral, but helps limit the sources of pollution. It is much more easier to manage the pollution produced from one power plant, than that from thousands upon millions of vehicles.


\subsubsection{Rising Fuel Prices}

Public concern over rising fuel prices has also sparked interested in electric vehicles. As the vehicles do not consume traditional fuels directly, and use a cheaper source of power, they become attractive to those trying to save money. This has created consumer demand for this mode of transport, triggering research into the efficiency and effectiveness of said vehicles.


\section{Hardware used} % section headings are printed smaller than chapter names
% intro


\subsection{Vehicle used in this project}

The vehicle that this project will be developed for is a 2008 Hyundai Getz. This vehicle contains a 39kW electric motor \cite{rev_eco}. To power these motors, batteries have been installed resulting in a providing a total of 13kWh of power. These batteries are protected by the use of an EV Power Battery Balancing System \cite{evpowerbms}.  This system balances the charge contained in each cell, and prevents the individuals cells from under or over charging.

\subsection{Embedded Controller}

An Eye-bot M6 embedded controller is the platform that all software developed in this project will run on. This controller contains a Gumstix board with a 400Mhz PXA255 processor that uses the ARMv5TE instruction set \cite{blackham_thesis}. This runs a version of busy-box linux and has also had libraries developed by the University of Western Australia to access a Xilinx FPGA and a Samsung LTE430WQF0 touchscreen \cite{macleod_thesis}. This hardware is no longer supported by the manufacturer, and as of writing the kernel is a few years old \cite{gumstix_buildroot}. This hardware is currently installed in the vehicle.

\subsection{Battery Monitor}

The battery monitor is current installed in the car also. It is a commercial product manufactured by TBS electronics \cite{tbs_features}. This device is able to read the voltage charge and current of the battery cells installed in the vehicle. It will interface with the embedded controller via a RS232 serial link. It transmits all the information automatically over the link everyone second \cite{tbs_specs}.

\subsection{GPS}

GPS works via the concept of resection.  This is a mathematically concept that can determine the an objects current location relative to 3 or more other locations \cite{intro_to_gps}. There are many satellites in the sky which transmit signals received by a GPS unit. By calculating the intersection of 3 or more spheres, each centered at a different satellite, the unit is able to calculate its current position on the earth. Once positional data has been obtained this can be compared against other values such as time, in order to calculate distance. This information can also be interpreted to calculate the distance the vehicle has traveled over arbitrary intervals.

\section{Technological concepts}

\subsection{Threading and Process}

Having a program complete multiple tasks at once is a must for modern day systems. This is usually achieved through two methods. The first method is threading, in which a single program will have multiple "threads" accessing the same memory. The other technique is daemons, or individual processes, in which each concurrent aspect of the system will be handled by a separate program.

\subsection{Messaging}

Communication between devices can often be abstracted to the use of messages. There are two ways of dealing with messages, the two methods are called broker and broker-less \cite{zeromq_broker} . In this project a broker-less model will be used.

\subsubsection{Brokerless}

The broker less method allows each component to connect directly to any other component. There is no centralized "broker" to send messages to. This provides extra resilience as there is no single point of failure. A open source implementation that supports the broker-less method is called ZeroMQ \cite{zeromq_guide}. It has many different internal models and supports communication over TCP/IP.






% dextran, starch, glycogen
%Most organisms use polymers of glucose units for energy storage and differ only slightly in the way they link together monomers to sometimes gigantic macromolecules. Dextran of bacteria %is made from long chains of $\alpha$-1,6-linked glucose units. 

%: ----------------------- HELP: special characters
% above you can see how special characters are coded; e.g. $\alpha$
% below are the most frequently used codes:
%$\alpha$  $\beta$  $\gamma$  $\delta$

%$^{chars to be superscripted}$  OR $^x$ (for a single character)
%$_{chars to be suberscripted}$  OR $_x$

%>  $>$  greater,  <  $<$  less
%≥  $\ge$  greater than or equal, ≤  $\ge$  lesser than or equal
%~  $\sim$  similar to

%$^{\circ}$C   ° as in degree C
%±  \pm     plus/minus sign

%$\AA$     produces  Å (Angstrom)




% dextran, starch, glycogen continued

%Starch of plants and glycogen of animals consists of $\alpha$-1,4-glycosidic glucose polymers \{lastname07}. See figure \ref{largepotato} for a comparison of glucose polymer structure and chemistry. 
%
%Two references can be placed separated by a comma \cite{lastname07,name06}.

%: ----------------------- HELP: references
% References can be links to figures, tables, sections, or references.
% For figures, tables, and text you define the target of the link with \label{XYZ}. Then you call cross-link with the command \ref{XYZ}, as above
% Citations are bound in a very similar way with \cite{XYZ}. You store your references in a BibTex file with a programme like BibDesk.





%\figuremacro{largepotato}{A common glucose polymers}{The figure shows starch granules in potato cells, taken from %\href{http://molecularexpressions.com/micro/gallery/burgersnfries/burgersnfries4.html}{Molecular Expressions}.}

%: ----------------------- HELP: adding figures with macros
% This template provides a very convenient way to add figures with minimal code.
% \figuremacro{1}{2}{3}{4} calls up a series of commands formating your image.
% 1 = name of the file without extension; PNG, JPEG is ok; GIF doesn't work
% 2 = title of the figure AND the name of the label for cross-linking
% 3 = caption text for the figure

%: ----------------------- HELP: www links
% You can also see above how, www links are placed
% \href{http://www.something.net}{link text}

%\figuremacroW{largepotato}{Title}{Caption}{0.8}
% variation of the above macro with a width setting
% \figuremacroW{1}{2}{3}{4}
% 1-3 as above
% 4 = size relative to text width which is 1; use this to reduce figures



%
%Insulin stimulates the following processes:
%
%\begin{itemize}
%\item muscle and fat cells remove glucose from the blood,
%\item cells breakdown glucose via glycolysis and the citrate cycle, storing its energy in the form of ATP,
%\item liver and muscle store glucose as glycogen as a short-term energy reserve,
%\item adipose tissue stores glucose as fat for long-term energy reserve, and
%\item cells use glucose for protein synthesis.
%\end{itemize}

%: ----------------------- HELP: lists
% This is how you generate lists in LaTeX.
% If you replace {itemize} by {enumerate} you get a numbered list.


 


%: ----------------------- HELP: tables
% Directly coding tables in latex is tiresome. See below.
% I would recommend using a converter macro that allows you to make the table in Excel and convert them into latex code which you can then paste into your doc.
% This is the link: http://www.softpedia.com/get/Office-tools/Other-Office-Tools/Excel2Latex.shtml
% It's a Excel template file containing a macro for the conversion.

%\begin{table}[htdp]
%\centering
%\begin{tabular}{ccc} % ccc means 3 columns, all centered; alternatives are l, r
%
%{\bf Gene} & {\bf GeneID} & {\bf Length} \\ 
%% & denotes the end of a cell/column, \\ changes to next table row
%\hline % draws a line under the column headers
%
%human latexin & 1234 & 14.9 kbps \\
%mouse latexin & 2345 & 10.1 kbps \\
%rat latexin   & 3456 & 9.6 kbps \\
%% Watch out. Every line must have 3 columns = 2x &. 
%% Otherwise you will get an error.
%
%\end{tabular}
%\caption[title of table]{\textbf{title of table} - Overview of latexin genes.}
%% You only need to write the title twice if you don't want it to appear in bold in the list of tables.
%\label{latexin_genes} % label for cross-links with \ref{latexin_genes}
%\end{table}



% There you go. You already know the most important things.


% ----------------------------------------------------------------------



