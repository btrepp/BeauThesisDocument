\section{The REV Group}

The Renewable Energy Vehicle project aims to explore the use of electric engines in vehicles in place of the traditional internal combustion engine. This can provide many benefits; including helping remove societies dependency on oil and reducing emissions from personal transportation. The Hyundai Getz is designed to be a substitute for economical cars. This car is designed to fit into the “daily driver” market. This market needs smaller cars, with short range but cheap running costs, as well as being enjoyable to drive. 

\subsection{Motivations} 
 
Many factors motivate research into electric based vehicles. The two most prominent are the emissions of traditional vehicles and the continuing rise in cost of fuels.

\subsubsection{Pollution} % subsection headings are again smaller than section names
% lead

Electric vehicles are advantageous over traditional ICE vehicles as they operate with zero emissions while operating . These vehicles have no exhaust, so therefore have no emissions. While this does not make them completely pollutant free, it does help limit and control the emissions being produced by the act of transport. It is important to remember when discussing electric vehicles that the components must be manufactured using industrial processes and the act of generation electricity. This does not making them truly carbon neutral, but helps limit the sources of pollution. It is much more easier to manage the pollution produced from one power plant, than that from thousands upon millions of vehicles.


\subsubsection{Rising Fuel Prices}

Public concern over rising fuel prices has also sparked interested in electric vehicles. As the vehicles do not consume traditional fuels directly, and use a cheaper source of power, they become attractive to those trying to save money. This has created consumer demand for this mode of transport, triggering research into the efficiency and effectiveness of said vehicles.
