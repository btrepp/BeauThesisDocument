% this file is called up by thesis.tex
% content in this file will be fed into the main document

%: ----------------------- name of chapter  -------------------------
\chapter{System Design} % top level followed by section, subsection
\label{sec:design}

%: ----------------------- paths to graphics ------------------------

% change according to folder and file names
\ifpdf
    \graphicspath{{4/figures/PNG/}{4/figures/PDF/}{4/figures/}}
\else
    \graphicspath{{4/figures/EPS/}{4/figures/}}
\fi

%: ----------------------- contents from here ------------------------

\section{Distributed Design}

The system design developed is a distributed system. This means all that all the components of the system are isolated from each other. They are able to run independently, and can developed separately. In order to share data between different components of the system, a protocol is developed using the ZeroMQ messaging framework to transmit the data between components.

\subsection{Isolation}

The main motivation for developing the complete system as a distributed system is isolation. Each component is isolated from the code in other components. This isolation property brings numerous advantages to the system design. 

The first advantage is that each component of the system can be developed independently. As the code for each component is separate, the component can be developed at different rates, even completely different developers. This can be used to speed up the development, and debugging process, as individuals do not need to worry about conflicts created due to changes they may make.

Another advantage of the isolation property is robustness. As each system component is run, it consumes it's own space of the operating systems memory. This space is protected from interference by the operating system. No other running process will be able to manipulate it's contents. This protects the process as errors in other components cannot affect this process. If any process were to hang, or exhibit unexpected behavior, it's effects will be isolated to the process that caused it. This provides stability in the final system, as an error in one component will still allow all the other components to function correctly.

\subsection{Distribution of Components}

Another advantageous property of the distributed system is that each component can run on isolated hardware. The entire system can be spread among many devices, or just the one. This property allows for increased flexibility in the allocation of resources. If the current hardware is unable to support all the components at the same time, extra hardware can be purchased to take some of the load. Due to the design, the displaced components will not need to be re-compiled or modified in order to run on the new hardware. The displaced components will only need to be installed and configured to communicate with the existing hardware. 

By allowing flexibility in the layout of hardware, the system can be simply expanded in future. Hardware requirements can be much more flexible, allowing older slower hardware to be supplemented by newer components, rather than replaced entirely. This will save costs in development of the system.

This property also has advantages for debugging the system. Components can be run on other devices, such as the machines used to developed them. These machines can also use the network protocol to communicate to the existing system. This allows utilities to simulate the functionality of components, see Appendix \ref{app:utilities}, which greatly simplifies the task of debugging the system.

\subsection{Simpler Component Architecture}

As the system uses individual processes for each component, development does not require extensive knowledge of threading. Threading and synchronization is complicated. It is hard to analyze and debug multi-threaded programs, as any variable may be changing at any time. This architecture removes threads, as concurrency is provided by running multiple components in isolated processes. This greatly simplifies the design and debugging of each individual component, which results in a more stable and faster developed system.

\section{Network Protocol}

As each component cannot access the memory of any other component, a protocol was developed to facilitate communication between different components. This protocol is transmitted using the ZeroMQ layer, which exists on top of the standard TCP/IP network layer. This library was used as it has been developed by a larger group of developers than this single project, and as such will implement more functionality and have more robustness than could have been developed in the time-frame for this project.

\figuremacro{zeromqnetwork}{ZeroMQ layer in network topology}{}

Figure \ref{zeromqnetwork} shows the ZeroMQ layer with respect to the application and network layers. It illustrates how ZeroMQ exists between the applications and the TCP/IP layer. ZeroMQ can also be used to communicate without using the network, however both applications must be running on the same device. This gives performance increases, as it does not transmit the messages onto the network.

\subsection{Messages}

The ZeroMQ layer transmits all information as messages, as opposed to the standard networking design or streams. Figure \ref{messagesvstreams} shows the difference between these two methodologies.

\figuremacro{messagesvstreams}{Messages Vs Streams of Data}{}

TCP transmits data as a stream, it is stored and then read from a buffer on the receiving end. This method of transmission makes it hard to synchronize were the data stream has begun. Assumptions must be made as to the position of the currently recieved byte in the stream. While the nature of TCP does provide some  guarantees as to the start of the stream, a careless error in reading the length of the message could result in the program believing the wrong location is the current one. This can cause instability in the program, as the data would be interpreted incorrectly. This can either occur via accidental, or malicious means, such as a buffer overflow attack \cite{bufferoverflow}. 

A ZeroMQ message differs from a TCP stream as it is a bounded element of data. A message is received in full when it arrives. This allows the entire message to be processed, and removes concern about missing data that may still be on the network.  It is important to note that the TCP issues still exist over the ZeroMQ layer, as ZeroMQ runs on-top of the TCP layer. It is still theoretically vulnerable to these issues, but these problems can be prevented by different programming techniques. As these protections exist in this layer, there is no concern with implementing them, and the layer can be used in multiple projects, simplifying development time.

\subsection{Description of Network Protocol}

As the system communicates via messages, no synchronization is needed in order to sync the sender and receiver. The protocol is thus a standard ordering of bytes that can be used to identify different messages from each other. Table \ref{tab:networkprotocol} shows the meaning of each byte.

\begin{table}
    \begin{tabular}{*{5}{|l}|}
        \hline
        Byte 1 & Byte 2 & Byte 3 & Byte 4 & Byte 5  \\ \hline \hline
        \multicolumn{5}{|c|}{ Header}  \\ \hline
	\multicolumn{3}{|c|}{Filter} & Minor Filter & Revision Number \\
        \hline
    \end{tabular}
 \begin{tabular}{*{7}{|l}|}
        \hline
        Byte 6 & Byte 7 & Byte 8 & Byte 9 & Byte 10 & Byte 11 & Byte ... \\ \hline \hline
       \multicolumn {7}{|c|}{Data Payload} \\ \hline
 \multicolumn {7}{|c|}{Meaning depends on filter} \\
        \hline
    \end{tabular}
	\caption{Network protocol for the distributed system}
	\label{tab:networkprotocol}
\end{table}

Each message is started with a 5 byte header, followed by a variable length data payload. The formatting of the data inside the payload depends on the individual message being transmitted. The header consists of a 3 byte major filter, a single minor filter, and the revision number of the protocol. As the major filter takes up 3 bytes there are over 16 million possible unique messages, see equation \ref{eq:headerpossibilities}. This is more than will ever be needed, however it provides a 3 byte way of identifying messages. This is useful for inspecting the bytes, as it is easy to remember 3 unique numbers and differentiate between them. 

\begin{align}
\label{eq:headerpossibilities}
\mathrm{Unique Possible Messages} &=\mathrm{Possible\ values\ of\ a \ byte}^3 \\
&= 2^8*2^8*2^8 \notag \\
&= 16777216 \notag
\end{align}

After the major filter is a minor filter. The purpose of this filter is to differentiate between subsets of the same component. If a component wishes to perform some form of processing on the data, or to transmit its message in a different format, it can specify a minor filter to use. This differentiates the two different messages, but groups them logically as coming from the same source.

The last component of the header is the revision number. This is used to break compatibility with older programs. Whatever component is receiving the messages will only process messages that have a compatible revision number. When a component has a major change in the structure of it's messages, it will increase the revision number transmitted. This will cause any other components expecting the older revision to not accept the new version. This helps avoids errors, as the old receiver will not mis-interpret the new message, it will ignore it and record an error.


\section{Error Logging}
\label{sec:dmesglogging}

All components of the system will log any errors or unusual occurrences so they can be examined and the causes investigated. There are many different methods that can be utilized in order to log this data. The simplest is to allow each daemon to create it's own file to log to. This method is simple to implement, but can become difficult to manage. Code must be written in each module to record the error messages, which requires time and effort to debug. The files that the system is logging to must also be managed, as there is only finite space available on the device.

As all components of the system will be run on Linux, a much more elegant solution exists to solve this problem. A set of standards was defined for logging data from programs. This method is commonly referred to as "syslog" \cite{syslog}. These logging mechanisms are native to most Linux installations, they exist on the platform the device is running on. There have been implementations of syslog on Linux, Unix and BSD platforms. Syslog separates the logging functionality from the code of the program, which helps simplify development of the program. It has been developed by many contributors over many years \cite{syslog}. This makes it much more robust and scalable than any logging system that could have been developed for this thesis.

The added advantage of using syslog is that it has been developed to also work over a network. A Log device can log messages to a Log collector that is running on a different device. This provides a good synergy with the system design. As the logging of each device can take place on one central device, making it easier to check the log files for any errors.

All the components developed in this project use the in syslog.h file which is a part of the standard C libraries. There is no special libraries required to utilize the functions required to log to the system logger. By default the log messages are stored in \emph{/var/log/messages}. The resulting log messages contain information about the process that created the message, the time it occurred, and the contents of the message itself. This allows the log files to be filtered for specific circumstances, helping track down errors that may occur in the system.



%: ----------------------- contents from here ------------------------

\section{GPS Module}

\subsection{Hardware}

A vital part of the data logging and user-interface of the software is finding the cars current location. This is done by the use of a “off-the self” GPS unit. Currently the system is using a Qstar usb equipped GPS receiver. This receiver operates at a rate of 10hz , though it can be set to operate at a slower frequency of 1hz. For the purposes of recording positional data, along with estimating the vehicles current speed, the unit should run as fast as possible. The extra precision is useful for the data-logging aspect, with no negative effects on the user-display aspect.  

The GPS device can be enumerated as a standard serial port. This is beneficial as it can be used on any device that has the correct drivers and a available usb port. As it appears as a normal serial port, it can be queried using standard system routines. This allows the program that reads the device to operate any custom knowledge of the device it is connected to, aside from the serial parameters to make the connection. 

\subsection{ Drivers}
\label{sec:gpsdrivers}

While the Eyebot M6 has hardware usb support, it was not immediately compatible with the GPS sensor. Various versions of usb-serial drivers where tried each with their own problems. The main cause of this difficultly was the out-dated Linux kernel being run in the system. This was kernel version number 2.6.17 and was released in 2006, which is 5 years old as of writing \cite{osnews}. This was a major cause of incompatibilities, as the GPS receiver was manufactured a significant time after this kernel was written. The drivers had no clue as to what the usb product keys were, nor the specific quirks that the devices may have had.

The first driver attempted was the generic usb-serial driver, included as a kernel module in 2.6.17. This drivers success would mean that the sensor and program could be easily installed in just about any machine running Linux. The driver would have matured after the 2.6.17 kernel, and newer kernels would have support by default. This is beneficial to the system as it would require the least amount of configuration and setup if the GPS program was set to run in a different machine.

Sadly this driver did not perform correctly with the GPS device. While the driver was able to be loaded into the kernel without any errors, it caused problems when trying to associate with the GPS. The device uses extra features, which were unsupported by the generic driver. This caused strange symptoms in the operating system. The main symptom of an incorrect driver was the generation of the /dev/ttyUSB0 device. This availability of this device implies that a tty is available to read/write from. Due to the incompatibility of the driver, this serial port would never report any bytes to be read, which is why it is unsuitable for use with this device. Customizing the generic driver to support this device would be unfeasible because it is unlikely that newer versions of this driver would support the device. This leads to the situation that if the GPS program is ported to a different machine, a custom version of the generic usb serial drivers would have to be ported as well.

As the GPS device did not work with the generic serial drivers, alternative drivers were investigated in order to support this device. Experiments indicated that this device was automatically detected and loaded in a newer kernel. This functioned correctly and was able to communicate with the GPS device at the full rate of 10hz. The driver used by this kernel was called cdc-acm. Further investigation showed that this driver could be included as a kernel module for the gumstix platform.

This driver was not immediately compatible with the device. This was because the device was manufactured after the kernel . As such the driver did not recognize the manufacturer ID and product ID of the GPS. The driver was then modified to include this information and re-deployed to the eye-bot. This was successful in creating the virtual serial device inside /dev, and also in allowing data to be read from this device.

While the cdc-acm driver was able to be loaded and functioned, it still contained errors. If the system was under intense CPU load, the program may not run quick enough to remove all the data from the serial port buffer. This would cause the operating system to throttle the port. Examination of the driver source code reveals that new information is dropped while this driver is throttled. This is acceptable behavior in this instance, however this driver had a race condition. If the TTY was throttled under certain conditions, it would be unable to unthrottle later on. This cause the TTY to drain its buffer and never accept any new data from the GPS even if its buffer was empty. The driver was further modified to include spin-locks, a primitive kernel locking technique, in order to prevent this situation. The driver is now able to run for extended periods of time without locking up, enabling a reliable GPS reporting mechanism to be developed. 

\subsection {Design}

Development of the GPS reporting component was done in C. This was chosen as it is a relatively low level language, with wide support. It is simpler to understand than more complicated object oriented style languages. This makes it a good choice for the GPS reporting mechanism, as it only has to do one task. In order to ensure that the code can be easily modified by future programmers, the structure of this program is simple. It runs in only one-thread, aside from the back ground ZeroMQ threads, and thus requires no concurrency management.

\subsubsection{Initial Design}

This first iteration of this code used blocking ports and read a single byte at a time. This was the style used to match existing examples \cite{thesis_varma}. This was a functional design however it did lead to some problems. One problem with this approach was excessive throttling. The process would be woken up every time one character could be read, and would only remove one character from the buffer, even if there were hundreds waiting. This is a bad situation, as the process will continually be awoken to do trivial work. This steals cpu-time from another process, and caused the program to appear “sluggish”. This design was also in-sufficient as blocked the process while attempting to read from the port. This would cause the program to appear to hang if no data was available. It made it difficult to diagnose errors with this program.

\subsubsection{Final Design}

The design was refined in order to support bulk-reads and non-blocking operation. This fixes the two problems with the first approach. Rather than reading one character at a time, the program now reads as many as possible and stores the information into it's own circular buffer. This allows the serial port to be purged as quickly as possible. It also has the added feature of allowing the program to decide how to discard messages in the case that it cannot keep up with the GPS. Figure \ref{gpsflowchart} shows the program flow of the final design.

\figuremacro{gpsflowchart}{Flow chart of the GPS daemon}{}


\subsubsection{Network Protocol}

Table \ref{tab:gpsprotocol} shows the protocol for the GPS message when transmitted over the network. The protocol uses a binary format instead of an ASCII based one. This reduces the space/data transmitted over the link, which helps reduce cost and improve speed. The motivation for ASCII based protocols is that control characters can be used to help synchronize the data. As this design uses ZeroMQ in order to manage the flow of data, such control characters are unnecessary. All values are transmitted in network order, this is big-endian order so that the most significant byte is transmitted first.

\begin{table}
    \begin{tabular}{*{6}{|l}|}
        \hline
        47 &  50 & 53 & 0 & 0  & 01    \\ \hline \hline
        \multicolumn{5}{|c|}{ Header} & \multicolumn {1}{|c|}{Data Payload} \\ \hline
	\multicolumn{3}{|c|}{Filter} & Minor Filter & Revision Number & Fix \\
        \hline
    \end{tabular}
\end{table}

\begin{table}
    \begin{tabular}{*{6}{|l}|}
        \hline
        4E1FE968 & 00000000 & C1FFD54D & 42E7A1F1 & 43B40000 & 41200000 \\ \hline \hline
         \multicolumn {6}{|c|}{Data Payload} \\ \hline
	Time(s) & Time(us) & Latitude & Longitude & Bearing & Speed \\
        \hline
    \end{tabular}
	\caption{Network protocol for GPS daemon}
	\label{tab:gpsprotocol}
\end{table}



% ---------------------------------------------------------------------------
%: ----------------------- end of thesis sub-document ------------------------
% ---------------------------------------------------------------------------



%: ----------------------- contents from here ------------------------

\section{TBS Module}

\subsection{Hardware}

The most important external device used in the user interface and data-logging aspect of the software is that of the battery monitoring module. The car has 45 Lithium Ion batteries installed, and it is useful to monitor the charge, current and voltage of the battery cells at all times. The system that the monitoring software runs on is not a highly reliable embedded system. It requires a few minutes to start up, and consumes too much power to leave running all the time. As such a different device is used to track the health and charge of the batteries. This device is a e-xpert pro battery monitor manufactured by TBS electronics. This is a commerical unit which increases the reliability of the data that it produces. Unlike the eyebot,  it is powered as long as the cells in the car remain energized, so it will always log and monitor the health of the batteries.

\subsection{E\-xpert protocol}

The e-xpert device has a set protocol that it uses to communicate with other devices. It uses a RS232 connection over a 9 pin plug. This is a common way of communicating with external modules, and the eyebot has a serial port available to communicate with the e-xpert pro module. The module communicates using asynchronous communication. It automatically sends out updates at a rate of 1hz\cite{e_xpert}. These updates contain all the information that is recorded by the e-expert pro module. As the communication is asynchronous, this will happen automatically, even while the eyebot is not connected. This is not a problem as the e-xpert pro does not expect a response. This mode of operation is referred to as broadcast mode in the e-xpert documentation \cite{e_xpert}.

\begin{table}
\begin{center}
    \begin{tabular}{|l|l|}
        \hline
        Property & Value \\ \hline
        Baud Rate    & 2400 \\
        Data Bits    & 8    \\ 
        Stop Bits    & 1    \\ 
        Parity       & Even \\ 
        Flow Control & None \\
        \hline
    \end{tabular}
	\caption{Connection settings for Battery Monitor}
\end{center}
\end{table}


\subsubsection{Destination and Start Byte}
\label{sec:expertprotocol}

The message data that the module outputs is transmitted via serial. The first byte in the message is the start byte, in order to identify this start byte as the start byte it must be unique and never occur anywhere inside the payload. This is done by reserving the most significant bit (MSB) to be one only if it is a start or ending byte. The documentation refers to this bit as the IDHT (Identify Header Trailer ) bit. This does mean that there are only 7 bits available in each byte for transmitting data, but guarantees that the start and end of messages can be synchronized. As the first bit is a one due to the start byte being the header, the value of this byte is greater than 0x80. The rest of the bits in this first byte are the destination address. While communicating with a PC in broadcast mode, these bits can be ignored \cite{e_xpert}. The module also will not know where it is sending the byte, it is in broadcast mode, so there is no destination address. Thus the destination address bits will be 0, so the first header byte is always received as 0x80.

\subsubsection{Source}

The next byte transmitted is the source address. This byte is not a IDHT byte, so the MSB will always be 0. The device installed in the car, "e-xpert pro" will always set the source address as being 0. Combining this with the IDHT bit results in the second byte always being hex 0.

\subsubsection{Device ID}

The third byte in the message is the device ID. This is a unique number that identifies the type of equipment being used. This number is set by the manufacturer to distinguish different devices it it's product range. For the case of this product, the "e-xpert pro" the device id is 0x22.

\subsubsection{Message Identifier}

The e-xpert pro module transmits a variety of messages, which can be categorized into three groups. These different groups are handshake, commands and data. In broadcast mode, handshake and command messages are not required in order to extract information from the battery monitor. Table \ref{tab:tbsmessages} shows the hexadecimal values for different messages.

\begin{table}
\begin{center}
    \begin{tabular}{|l|l|}
        \hline
        Property & Value \\ \hline
        Battery Voltage   & 0x60 \\
        Battery Current    & 0x61    \\ 
        Amphours    & 0x62    \\ 
        Charge       &  0x64 \\ 
        Time Remaining & 0x65 \\
        \hline
    \end{tabular}
	\caption{Hexadecimal values of different TBS messages}
	\label{tab:tbsmessages}
\end{center}
\end{table}

\subsubsection{Data}

Following the message identifier is the actual payload of the message. This can take on various forms, but in the most simplistic sense is a number spread across a few bytes. Figure \ref{tbsdatafigure} shows the data layout for the battery voltage message. For more information about the different values, see "e-xpert pro communications interface specifications" \cite{e_xpert}.  Due to the MSB of each byte being reserved, it is only possible to transmit 7 bits of data inside a byte, any value that requires more than 7 bits to be represented must be transmitted across multiple bytes.  

\figuremacro{tbsdatafigure}{Structure of data payload}{Note that each byte transmitted only has seven usable bits of data, the 8th bit becomes the lowest bit in the next byte \cite{e_xpert}}

\subsubsection{Trailing Byte}

The last byte in the message is the "end of transfer" byte \cite{e_xpert}. The purpose of this byte is to signal that the message has been sent. Like the starting byte, the MSB of this byte is set to one. The rest of the bits in this byte are also set to one, to signify that this is the end byte, rather than the start byte. The value transmitted is 0xFF, and this is the only location in which 0xFF can appear.


\subsection {Design}

Like the GPS module, the battery monitor module was developed using C. This keeps in line with the design principles of making each component as simple as possible. Figure \ref{tbsflowchart} shows the program flow of this daemon.

\figuremacro{tbsflowchart}{Flow chart of the battery monitor daemon}{}

\subsubsection{Network protocol}

Table \ref{tab:tbsprotocol} shows the protocol for the battery monitor message when transmitted over the network. As the design uses a binary format file like the GPS module, this format is impossible to read natively, but more efficient on space. There is no need for these values to be read on the network layer anyway. All values are transmitted in big endian network order.

\begin{table}
    \begin{tabular}{*{7}{|l}|}
        \hline
        54 &  42 & 53 & 0 & 0  & 4e1fe968  &         00000000   \\ \hline \hline
        \multicolumn{5}{|c|}{ Header} & \multicolumn {2}{|c|}{Data Payload} \\ \hline
	\multicolumn{3}{|c|}{Filter} & Minor Filter & Revision Number &Time(s) & Time remaining (s) \\
        \hline
    \end{tabular}
%\end{table}
%
%\begin{table}
    \begin{tabular}{*{6}{|l}|}
        \hline
 42b40000 & 43710000 & c150000 &  00000000 & 00000000 & 00000000 \\ \hline \hline
         \multicolumn {6}{|c|}{Data Payload} \\ \hline
	  Charge(\%) & Voltage(V) & Current(A) & AmpHours(Ah) & Temperature(C) & Status \\
        \hline
    \end{tabular}
	\caption{Network protocol for Battery Monitor daemon}
	\label{tab:tbsprotocol}
\end{table}



% ---------------------------------------------------------------------------
%: ----------------------- end of thesis sub-document ------------------------
% ---------------------------------------------------------------------------



%: ----------------------- contents from here ------------------------

\section{Trip Meter}



% ---------------------------------------------------------------------------
%: ----------------------- end of thesis sub-document ------------------------
% ---------------------------------------------------------------------------



%: ----------------------- contents from here ------------------------

\section{Interial Measurement Unit Display Panel}

Developed in section 


\figuremacro{imu}{IMU display panel}{This figure shows the information output by the IMU daemon }



% ---------------------------------------------------------------------------
%: ----------------------- end of thesis sub-document ------------------------
% ---------------------------------------------------------------------------



%: ----------------------- contents from here ------------------------

\section{File Logger}
\label{sec:filelogger}

The previously mentioned components all generate data at various periodic rates. Data acquisition is useful, but it would be more useful to record this data so it can be analyzed at a later stage. This section outlines the design of a logging mechanism that will record the Battery Monitor and GPS data to a file. 

\subsection {Design}

\subsubsection{Program Flow}
The process flow of the logger component is shown in Figure \ref{loggerflowchart}. This process will log the data for bother the GPS and Battery monitor components. It adapts to the speed of the slowest component. This was done to make the processor utilization more efficient. As the system will only write to a file when both values have data, it will automatically match that of the slowest component. Thus the limitation in the resolution that can be logged, is the resolution of the slowest data generating component.

\figuremacro{loggerflowchart}{Flow chart of the Logger component}{This figure shows the process flow of the logger daemon. It will log both the battery monitor and GPS messages together, or individually one component is not transmitting.}

If a component is does not transmit any messages in a reasonable time, the system will log the other values. This has been implemented so the system will not wait forever, for values that will never come. This allows the battery health, or the current location of the car to be recorded in all instances. Ideally this should never occur, but if the vehicle loses GPS signal, for instance when inside a tunnel, it is desirable to still record any changes in the battery status.

\subsubsection{File storage}

A massive file containing all the information logged is troublesome to manage. Each entry in the logger is recorded with a time-stamp. This time-stamp can be used to determine the current system date. A more manage-able solution to storing the data is to provide a unique file for each day. All new entries are appended to the end of this file. This makes it much easier to manage the logged data, and also makes it easier to observe how many days the logger has been active for.

\subsubsection{File Format}

The file format is a simple ASCII based CSV file. Each property of an entry is separated by a comma. A new entry occurs on the next line. This format is simple, human read-able and does not require much space to store. Figure \ref{loggingfilesample} shows a sample entry of the text file logger.

\figuremacro{loggingfilesample}{Sample entry in log file}{This figure shows a sample entry in the file logger. The file is CSV formatted. The second line shows what each entry in the file means}

As seen in figure \ref{loggingfilesample} each line in the file contains 78 characters. Each line will also store the newline character as-well. With the information the rate that the logging file consumes storage can be calculated using equation \ref{eq:logfilesize}. In ideal conditions the log file will fill up at a rate of 4760 bytes per minute.  This is equivalent to 277 kBytes an hour, or 6.5 megabytes per day. The internal storage of the eyebot is approximately 16 megabytes, but most of this needs to be taken up by the operating system and the code to run the various components. Luckily the eyebot can support usb storage devices.  A inexpensive 16GB usb drive has enough capacity to last 2520 days, which is more than enough logging capacity for the system.

\begin{align}
\label{eq:logfilesize}
\mathrm{Bytes\ per\ minute} &=\mathrm{Characters\ per\ line}* \mathrm{Frequency}*\mathrm{Seconds\ per\ minute}\\
&= 79*1*60 \notag \\
&= 4740 \notag
\end{align}

\section{Network Logger}

Previously developed by other students was a server that provides Internet logging functionality. This server exists on a separate system, and uses TCP/IP to communicate with various clients. The car is equipped with a 3g module that enables it to access this server remotely. This component is concerned with receiving information from the existing components, and logging them to the already existing server so they can be displayed and analyzed via a web-portal. The logger implements the protocol according to Pearce's thesis \cite{john_thesis}.

\subsection {Design}

An important limitation of any network communication is that the network is unstable. At any time the network connection could disappear, preventing the program from logging data to the server. This would cause the program to hang while it waits for communication to resume. In order to solve this problem non-blocking communication was used. Non-blocking communication is when any data transmission will return instantly from being called. This allows the program to examine what has happened and react accordingly.

If the network is down for an extended period of time, the system will not discard messages generated during this period of inactivity. These unsent messages are stored in memory in a queue. This queue maintains all the unsent messages, until the time in which they are able to be transmitted. Figure \ref{tcploggerflowchart} shows the program flow of the internet based logger.

\figuremacro{tcploggerflowchart}{Flow chart of the Network Logger component}{This figure shows the process flow of the logger daemon. It will log both the battery monitor and GPS messages together. This diagram has simplified the receiving of ZeroMQ messages}




% ---------------------------------------------------------------------------
%: ----------------------- end of thesis sub-document ------------------------
% ---------------------------------------------------------------------------





% ---------------------------------------------------------------------------
%: ----------------------- end of thesis sub-document ------------------------
% ---------------------------------------------------------------------------

