
%: ----------------------- contents from here ------------------------

\section{TBS Module}

\subsection{Hardware}

The most important external device used in the user interface and data-logging aspect of the software is that of the battery monitoring module. The car has 45 Lithium Ion batteries installed, and it is useful to monitor the charge, current and voltage of the battery cells at all times. The system that the monitoring software runs on is not a highly reliable embedded system. It requires a few minutes to start up, and consumes too much power to leave running all the time. As such a different device is used to track the health and charge of the batteries. This device is a e-xpert pro battery monitor manufactured by TBS electronics. This is a commerical unit which increases the reliability of the data that it produces. Unlike the eyebot,  it is powered as long as the cells in the car remain energized, so it will always log and monitor the health of the batteries.

\subsection{E\-xpert protocol}

The e-xpert device has a set protocol that it uses to communicate with other devices. It uses a RS232 connection over a 9 pin plug. This is a common way of communicating with external modules, and the eyebot has a serial port available to communicate with the e-xpert pro module. The module communicates using asynchronous communication. It automatically sends out updates at a rate of 1hz\cite{e_xpert}. These updates contain all the information that is recorded by the e-expert pro module. As the communication is asynchronous, this will happen automatically, even while the eyebot is not connected. This is not a problem as the e-xpert pro does not expect a response. This mode of operation is referred to as broadcast mode in the e-xpert documentation \cite{e_xpert}.

\begin{table}
\begin{center}
    \begin{tabular}{|l|l|}
        \hline
        Property & Value \\ \hline
        Baud Rate    & 2400 \\
        Data Bits    & 8    \\ 
        Stop Bits    & 1    \\ 
        Parity       & Even \\ 
        Flow Control & None \\
        \hline
    \end{tabular}
	\caption{Connection settings for Battery Monitor}
\end{center}
\end{table}


\subsubsection{Destination and Start Byte}
\label{sec:expertprotocol}

The message data that the module outputs is transmitted as shown in figure \ref{tbs_protocol}. The first byte in the message is the start byte, in order to identify this start byte as the start byte it must be unique and never occur anywhere inside the payload. This is done by reserving the most significant bit (MSB) to be one only if it is a start or ending byte. The documentation refers to this bit as the IDHT (Identify Header Trailer ) bit. This does mean that there are only 7 bits available in each byte for transmitting data, but guarantees that the start and end of messages can be synchronized. As the first bit is a one due to the start byte being the header, the value of this byte is greater than 0x80. The rest of the bits in this first byte are the destination address. While communicating with a PC in broadcast mode, these bits can be ignored \cite{e_xpert}. The module also will not know where it is sending the byte, it is in broadcast mode, so there is no destination address. Thus the destination address bits will be 0, so the first header byte is always received as 0x80.

\subsubsection{Source}

The next byte transmitted is the source address. This byte is not a IDHT byte, so the MSB will always be 0. The device installed in the car, "e-xpert pro" will always set the source address as being 0. Combining this with the IDHT bit results in the second byte always being hex 0.

\subsubsection{Device ID}

The third byte in the message is the device ID. This is a unique number that identifies the type of equipment being used. This number is set by the manufacturer to distinguish different devices it it's product range. For the case of this product, the "e-xpert pro" the device id is 0x22.

\subsubsection{Message Identifier}

The e-xpert pro module transmits a variety of messages, which can be categorized into three groups. These different groups are handshake, commands and data. In broadcast mode, handshake and command messages are not required in order to extract information from the battery monitor. Table \ref{tbs:messages} shows the hexadecimal values for different messages.

\begin{table}
\begin{center}
    \begin{tabular}{|l|l|}
        \hline
        Property & Value \\ \hline
        Battery Voltage   & 0x60 \\
        Battery Current    & 0x61    \\ 
        Amphours    & 0x62    \\ 
        Charge       &  0x64 \\ 
        Time Remaining & 0x65 \\
        \hline
    \end{tabular}
	\caption{Hexadecimal values of different TBS messages}
	\label{tab:tbsmessages}
\end{center}
\end{table}

\subsubsection{Data}

Following the message identifier is the actual payload of the message. This can take on various forms, but in the most simplistic sense is a number spread across a few bytes. Figure \ref{stuff} shows the data layout for the battery voltage message. For more information about the different values, see appendix \ref{APPENDIX}.  Due to the MSB of each byte being reserved, it is only possible to transmit 7 bits of data inside a byte, any value that requires more than 7 bits to be represented must be transmitted across multiple bytes.  


\subsubsection{Trailing Byte}

The last byte in the message is the "end of transfer" byte \cite{e_xpert}. The purpose of this byte is to signal that the message has been sent. Like the starting byte, the MSB of this byte is set to one. The rest of the bits in this byte are also set to one, to signify that this is the end byte, rather than the start byte. The value transmitted is 0xFF, and this is the only location in which 0xFF can appear.


\subsection {Design}

Like the GPS module, the battery monitor module was developed using C. This keeps in line with the design principles of making each component as simple as possible. Figure \ref{tbsflowchart} shows the program flow of this daemon.

\figuremacro{tbsflowchart}{Flow chart of the battery monitor daemon}{}

\subsubsection{Network protocol}

Table \ref{tab:tbsprotocol} shows the protocol for the battery monitor message when transmitted over the network. As the design uses a binary format file like the GPS module, this format is impossible to read natively, but more efficient on space. There is no need for these values to be read on the network layer anyway. All values are transmitted in big endian network order.

\begin{table}
    \begin{tabular}{*{7}{|l}|}
        \hline
        54 &  42 & 53 & 0 & 0  & 4e1fe968  &         00000000   \\ \hline \hline
        \multicolumn{5}{|c|}{ Header} & \multicolumn {2}{|c|}{Data Payload} \\ \hline
	\multicolumn{3}{|c|}{Filter} & Minor Filter & Revision Number &Time(s) & Time remaining (s) \\
        \hline
    \end{tabular}
%\end{table}
%
%\begin{table}
    \begin{tabular}{*{6}{|l}|}
        \hline
 42b40000 & 43710000 & c150000 &  00000000 & 00000000 & 00000000 \\ \hline \hline
         \multicolumn {6}{|c|}{Data Payload} \\ \hline
	  Charge(\%) & Voltage(V) & Current(A) & AmpHours(Ah) & Temperature(C) & Status \\
        \hline
    \end{tabular}
	\caption{Network protocol for Battery Monitor daemon}
	\label{tab:tbsprotocol}
\end{table}



% ---------------------------------------------------------------------------
%: ----------------------- end of thesis sub-document ------------------------
% ---------------------------------------------------------------------------

